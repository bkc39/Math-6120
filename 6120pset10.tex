\documentclass{article}
\usepackage[tmargin=1in,bmargin=1in,lmargin=1.5in,rmargin=1.5in]{geometry}
\usepackage{amsfonts,amsmath,amssymb,amsthm}
\usepackage{mathrsfs}
\usepackage{ccfonts}
\usepackage{relsize,fancyhdr,parskip}
\usepackage{graphicx}

\usepackage{tikz} \usetikzlibrary{shapes, shapes.geometric,
  shapes.symbols, shapes.arrows, shapes.multipart, shapes.callouts,
  shapes.misc,decorations.markings,decorations.shapes}

\pagestyle{fancy}
\lhead{Ben Carriel}
\chead{Math 6120 Problem Set 10}
\rhead{\today}

\headheight 13.0pt
\parskip 7.2pt
\parindent 8pt

\newcommand{\tab}{\hspace*{2em}}
\newcommand{\tand}{\tab\text{and}\tab}

\DeclareMathOperator{\N}{\mathbb{N}}
\DeclareMathOperator{\Z}{\mathbb{Z}}
\DeclareMathOperator{\Q}{\mathbb{Q}}
\DeclareMathOperator{\R}{\mathbb{R}}
\DeclareMathOperator{\C}{\mathbb{C}}
\DeclareMathOperator{\Ha}{\mathbb{H}}
\DeclareMathOperator{\D}{\mathbb{D}}
\DeclareMathOperator{\T}{\mathbb{T}}
\DeclareMathOperator{\capchi}{\raisebox{2pt}{$\mathlarger{\mathlarger{\chi}}$}}

\DeclareMathOperator{\divides}{\mathrel{|}}
\DeclareMathOperator{\suchthat}{\mathrel{:}}

\DeclareMathOperator{\lra}{\longrightarrow}
\DeclareMathOperator{\into}{\hookrightarrow}
\DeclareMathOperator{\onto}{\twoheadrightarrow}
\DeclareMathOperator{\bijection}{\leftrightarrow}
\DeclareMathOperator{\lap}{\bigtriangleup}

\newcommand{\problem}[1]{\noindent{\textbf{Problem #1}}\\}
\newcommand{\problempart}[1]{\noindent{\textbf{(#1)}}}
\newcommand{\exercise}[1]{\noindent{\textbf{Exercise #1:}}}

\newcommand{\der}[2]{\frac{\partial #1}{\partial #2}}
\newcommand{\norm}[1]{\|#1\|}
\newcommand{\lpnorm}[2]{\|#1\|_{L^{#2}}}
\newcommand{\diam}[1]{\text{diam}(#1)}
\newcommand{\seq}[2]{\{#1_{#2}\}_{#2 = 1}^\infty}

\newcommand{\conj}[1]{\overline{#1}}
\newcommand{\cis}[1]{\operatorname{cis}#1}
\newcommand{\res}{\operatorname{res}}

\newcommand{\real}{\mathrel{\text{Re}}}
\newcommand{\imag}{\mathrel{\text{Im}}}
\newtheorem*{thm}{\\ Theorem}
\newtheorem*{lem}{\\ Lemma}
\newtheorem*{claim}{\\ Claim}
\newtheorem*{defn}{\\ Definition}
\newtheorem*{prop}{\\ Proposition}

\begin{document}
\exercise{3.4.16}
\begin{enumerate}
\item To see that the Cauchy-Riemann operator
  \[
  \partial_{\bar{z}}\frac{1}{2}\left(\der{}{x} + i\der{}{y}\right)
  \]
  is elliptic we compute the characteristic polynomial
  \[
  P(i\xi) = \frac{1}{2}i\left(\xi_1 + i\xi_2\right)
  \]
  Then the triangle inequality immediately implies that
  \[
  |\xi| \leq |\xi_1| + |i\xi_2| = \frac{1}{2}|P(\xi)|
  \]
  which shows that $\partial_{\bar{z}}$ is elliptic.
\item To see that $1/\pi z$ is a fundamental solution of
  $\partial_{\bar{z}}$ first note that because $\frac{1}{|z|} \in
  L^1_{\text{loc}}$ the function $f(z) = 1/z$ defines a distribution
  on $\C \cong \R^2$. Moreover, by Theorem 2.9 we know that
  $\frac{1}{2\pi}\log|x|$ is a fundamental solution of $\Delta$ on
  $\R^2$. So for any $\varphi \in \mathcal{D}$ we see that
  \[
  \frac{1}{2\pi}\int \log|x|\Delta\varphi dxdy = \delta(\varphi) = \varphi(0)
  \]
  Because $\Delta = 4\partial_{\bar{z}}\partial{z}$ we see that
  \[
  \Delta (\frac{1}{2\pi}\log|x|)(\varphi) = 4\partial_{\bar{z}}
  (\frac{1}{2\pi}\log|x|)(\partial_{z}\varphi) = 4
  (\partial_{z}\frac{1}{2\pi}\log|x|)(\partial_{\bar{z}}\varphi)
  \]
  But the above is equivalent to the statement that
  \[
  (\frac{1}{\pi z})(\partial_{\bar{z}}\varphi) = \partial_{\bar{z}}
  \frac{1}{\pi z}(\varphi) = \delta(\varphi
  \]
  which means that $1/\pi z$ is a fundamental solution of
  $\partial_{\bar{z}}$.
\item Suppose that $f$ is continuous and $\partial_{\bar{z}}f = 0$ in
  the sense of distributions and let $C$ be a simple, closed curve in
  the plane. Because $\partial_{\bar{z}}$ is elliptic the
  distribution $U = \frac{1}{\pi z}\ast f$ satisfies
  $\partial_{\bar{z}} U = f$ by the previous part. We then apply
  Theorem 2.14 to the distribution $U$ to see that $U$ is in act a
  $C^{\infty}$ function that satisfies
  \[
  \partial_{\bar{z}}\int_C Udz = \partial_{\bar{z}}\int
  \frac{f(\zeta}{\zeta -z}dz = \int
  \frac{\partial_{\bar{z}}f(\zeta)}{\zeta -z}dz
  \]
  Then because $\partial_{\bar{z}} f = 0$ we have that
  \[
  \int_C fdz = \int_C Udz = 0
  \]
  So $f$ is analytic by Morera's Theorem. 
\end{enumerate}

\exercise{3.4.17}
\begin{enumerate}
\item To see that $\log |f(z)|$ is locally integrable we begin by
  recalling that because $f$ is meromorphic it can be written as the
  quotient of two entire functions $f(z) = g(z)/h(z)$. Then we have
  the identity
  \[
  \log |f(z)| = \log \left|\frac{g(z)}{h(z)}\right| = \log |g(z)| - \log |h(z)|
  \]
  Thus, it suffices to show that $\log |g(z)|$ is locally integrable
  when $g$ is entire.

  In the case that $g$ is an entire function then $g$ admits a
  factorization
  \[
  g(z) = z^m\prod_{i=1}^\infty E_n(z/a_n)
  \]
  where the $a_n$ are the roots of $g$ and the $E_n$ are the canonical factors
  \[
  E_n(z) = \log (1-z)\sum_{i=1}^\infty \frac{z^k}{k}
  \]
  Thus,
  \[
  \int \log \left|z^m\prod_{n=1}^\infty E_n(z/a_n)\right|dz \leq \int
  m + \sum_{n=1}^\infty \log |E_n(z/a_k)|
  \]
  So $\log|g(z)|$ is locally integrable if each of the $\log
  |E_n(z/a_n)|$ is. To verify this we expand
  \[
  \log |E_n(z/a_n)| = \log|
  (1-z/a_k)e^{\sum_{i=1}^\infty a_k^{-i}z^i/i} \leq \log |1-z_k/a_k| +
\sum_{i=1}^\infty \frac{z^i}{ia_k^i}
  \]
  We know that $\log |w|$ is locally integrable, so we only need to
  consider the integrability of the sum term. Only finitely many of
  the series above satisfy $|z^i/ia_k^i| > 1/2$ because the otherwise
  the sequence of roots would have a limit point and the function
  would be identically zero, which is clearly integrable. So for
  sufficiently large $k$ we have that
  \[
  \int |E_n(z/a_k)| \leq \int \log |1-z_k/a_k| + \int \sum_{i=k}^\infty 2^{-n}
  \]
  which is the sum of locally integrable functions and hence locally
  integrable. For large values of $z/a_k$ we will use the analytic
  continuation of the usual power series representation for $\log
  (1+z)$ yielding
  \[
  \log (1+z/a_n) + \sum \frac{z^k}{ka_n^k} = -\sum
  (-1)^n\frac{z^k}{ka_n^k} + \sum \frac{z^k}{ka_n^k} \leq \log |1+z/a_n|
  \]
  Which shows that
  \[
  \int |E_n(z/a_n)| \leq \int \log |1-z/a_n| < \infty
  \]
  for small roots $a_k$ and so each of the $E_n$ is locally integrable
  and we are done.
\item 
\end{enumerate}

\exercise{3.4.18}

The forward direction follows from the homogeneity condition because
we note that if we differentiate the identity $F_a(\varphi) =
a^\lambda f(\varphi)$ with respect to $a$ we get
\begin{align*}
  \der{}{a} F_a(\varphi) &= \der{}{a} a^\lambda f(\varphi) \\
  F(\der{}{a} \varphi^a) &= \lambda a^{\lambda -1}F(\varphi) \\
  F\left(\sum_{i=1}^n x_i \der{\varphi}{x_i}\right) &= \lambda a^{\lambda
      -1}F(\varphi)
\end{align*}
Then we observe that Linearity reduces the left side to
\[
F\left(\sum_{i=1}^n x_i \der{\varphi}{x_i}\right) = \sum_{i=1}^n x_i
F\left(\der{\varphi}{x_i}\right) = \left(\sum_{i=1}^n x_i
  \der{}{x_j}F\right)(\varphi)
\]
Setting $a = 1$ gives
\[
\sum_{i=1}^n x_i \der{}{x_j}F = \lambda F
\]
as desired.


\exercise{3.4.22}

First we will assume that $f \in L^1_{\text{loc}}$ and define $u(x,t) = f(x-t)$

\exercise{3.4.25}

In the forward direction, we suppose that $F$ is a positive
distribution. Then $F$ in particular is a positive linear functional
on $\R^d$. We then see that $F$ must in fact be a continuous function
because each $\varphi$ is supported on a compact set in $\Omega$ and
so $F$ is bounded hence, it is continuous. Then because $\mathcal{D}$
is dense in the space of continuous functions with compact support,
$C_0$, we apply the Hahn-Banach theorem to extend it to a continuous
function on $C_c$.

Now we need to show that the extension of $F$, call it $\tilde{F}$ is
positive. Consider a test function $\varphi$ convolved with a sequence
of mollifiers $\eta_n$. If we set $\varphi_n = \varphi \ast \eta_n$
then each of the $\varphi_n$ has compact support (because $\varphi$
does). Because each each of the above have compact support we see that
$\varphi_n \to \varphi$ uniformly and moreover that
\[
\tilde{F}\varphi = \tilde{F}(\lim_{n\to\infty} \varphi_n) =
\lim_{n\to\infty}\tilde{F}\varphi_n = \lim_{n\to\infty} F\varphi_n
\geq 0
\]
Where the second-to-last equality follows from density. So $\tilde{F}$
is positive and by the Riesz representation theorem we can find a
locally finite Borel measure $\mu$ such that 
\[
\tilde{F}(\varphi) = \int \varphi(x)d\mu
\]
And in particular we see that
\[
F(\varphi) = \tilde{F}(\varphi) = \int \varphi(x)d\mu
\]
as desired.

For the converse we simply note that a locally finite Borel measure $\mu$ is
a continuous function in $\Omega$ because it is bounded. Then we use
the fact that the test function $\varphi$ is non-negative and so the
distribution associated to the function $\mu$ is positive.

\exercise{3.4.27}
\begin{enumerate}
\item
\item
\item 
\end{enumerate}
\end{document}