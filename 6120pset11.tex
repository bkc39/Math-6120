\documentclass{article}
\usepackage[tmargin=1in,bmargin=1in,lmargin=1.5in,rmargin=1.5in]{geometry}
\usepackage{amsfonts,amsmath,amssymb,amsthm}
\usepackage{mathrsfs}
\usepackage{ccfonts}
\usepackage{relsize,fancyhdr,parskip}
\usepackage{graphicx}

\usepackage{tikz} \usetikzlibrary{shapes, shapes.geometric,
  shapes.symbols, shapes.arrows, shapes.multipart, shapes.callouts,
  shapes.misc,decorations.markings,decorations.shapes}

\pagestyle{fancy}
\lhead{Ben Carriel}
\chead{Math 6120 Problem Set 11}
\rhead{\today}

\headheight 13.0pt
\parskip 7.2pt
\parindent 8pt

\newcommand{\tab}{\hspace*{2em}}
\newcommand{\tand}{\tab\text{and}\tab}

\DeclareMathOperator{\N}{\mathbb{N}}
\DeclareMathOperator{\Z}{\mathbb{Z}}
\DeclareMathOperator{\Q}{\mathbb{Q}}
\DeclareMathOperator{\R}{\mathbb{R}}
\DeclareMathOperator{\C}{\mathbb{C}}
\DeclareMathOperator{\Ha}{\mathbb{H}}
\DeclareMathOperator{\D}{\mathbb{D}}
\DeclareMathOperator{\T}{\mathbb{T}}
\DeclareMathOperator{\capchi}{\raisebox{2pt}{$\mathlarger{\mathlarger{\chi}}$}}

\DeclareMathOperator{\divides}{\mathrel{|}}
\DeclareMathOperator{\suchthat}{\mathrel{:}}

\DeclareMathOperator{\lra}{\longrightarrow}
\DeclareMathOperator{\into}{\hookrightarrow}
\DeclareMathOperator{\onto}{\twoheadrightarrow}
\DeclareMathOperator{\bijection}{\leftrightarrow}
\DeclareMathOperator{\lap}{\bigtriangleup}

\newcommand{\problem}[1]{\noindent{\textbf{Problem #1}}\\}
\newcommand{\problempart}[1]{\noindent{\textbf{(#1)}}}
\newcommand{\exercise}[1]{\noindent{\textbf{Exercise #1:}}}

\newcommand{\der}[2]{\frac{\partial #1}{\partial #2}}
\newcommand{\norm}[1]{\|#1\|}
\newcommand{\lpnorm}[2]{\|#1\|_{L^{#2}}}
\newcommand{\diam}[1]{\text{diam}(#1)}
\newcommand{\seq}[2]{\{#1_{#2}\}_{#2 = 1}^\infty}

\newcommand{\conj}[1]{\overline{#1}}
\newcommand{\cis}[1]{\operatorname{cis}#1}
\newcommand{\res}{\operatorname{res}}

\newcommand{\real}{\mathrel{\text{Re}}}
\newcommand{\imag}{\mathrel{\text{Im}}}
\newtheorem*{thm}{\\ Theorem}
\newtheorem*{lem}{\\ Lemma}
\newtheorem*{claim}{\\ Claim}
\newtheorem*{defn}{\\ Definition}
\newtheorem*{prop}{\\ Proposition}

\begin{document}
\exercise{7.9.1}

To see that $f$ must be identically 0 in the polydisc $P_r(z^0)$ we
observe that $f$ has a power series representation
\[
f(z) = \sum_{|\alpha| \leq d} a_{\alpha}(z-z^0)^{\alpha}
\]
Moreover, we have a formula for the $a_{\alpha}$ given by Proposition
7.1.1 which states that
\[
a_{\alpha} = \frac{1}{(2\pi i)^n}\int_{C_r(z^0)} f(\zeta)\prod_{k=1}^d
\frac{d\zeta _k}{(\zeta_k - z_k^0)^{\alpha_k+1}}
\]
Then we use the fact that $f$ vanishes the polydisc
$\mathbb{P}_r(z^0)$ to conclude that $f$ must vanish in the disk
$|\zeta_1 - z_1^0| = r_1$ to see that
\[
a_{\alpha} = \frac{1}{(2\pi i)^n}\left(\int_{|\zeta_1 - z_1^0| = r_1}
  \frac{f(\zeta_1)}{(\zeta_1 -
    z_1^0)}\right)\left(\int_{C_r(z^0)}\prod_{k=2}^d \frac{d\zeta
    _k}{(\zeta_k - z_k^0)^{\alpha_k+1}}\right) = 0
\]
because the second term in the product is zero. Hence, $f$ is zero on
some neighborhood in $\mathbb{P}_r(z^0)$. Now we apply Proposition 7.1.2
to the functions $f$ and $0$ to conclude that they must agree on all
of $\mathbb{P}_r(z^0)$.

\exercise{7.9.2}
\begin{enumerate}
\item 
\end{enumerate}

\exercise{7.9.4}

\exercise{7.9.5}
\begin{enumerate}
\item
\item 
\end{enumerate}

\exercise{7.9.6}

Let $D_\delta(z) \subset \Omega$ be a disc centered at $z \in
\Omega$. We will apply Green's theorem (in the complex sense) to the
differential form
\[
\frac{F(\zeta) d\zeta}{\zeta - z}
\]
in the region $\Omega_{(\delta,z)} = \Omega \setminus
D_{\delta}(z)$. Indeed, Green's theorem states that
\[
\int_{\partial \Omega} \frac{F(\zeta) d\zeta}{\zeta - z} -
\int_{\partial D_{\delta}(z)} \frac{F(\zeta) d\zeta}{\zeta - z} =
\int_{\Omega_{(\delta,z)}} (\partial + \bar{\partial})\frac{F(\zeta)
  d\zeta}{\zeta - z}
\]
We then use linearity to expand the right hand side to
\[
\int_{\Omega_{(\delta,z)}} \partial\left(\frac{F(\zeta) d\zeta}{\zeta
    - z}\right) + \int_{\Omega_{(\delta,z)}}
\bar{\partial}\left(\frac{F(\zeta) d\zeta}{\zeta - z}\right)
\]
Then we note that
\[
\partial\left(\frac{F(\zeta) d\zeta}{\zeta - z}\right) =
\der{}{\zeta}\left(\frac{F(\zeta)}{\zeta - z}\right)d\zeta \wedge
d\zeta = 0
\]
Moreover,
\[
\bar{\partial}\left(\frac{F(\zeta) d\zeta}{\zeta - z}\right) =
\der{F}{\bar{\zeta}}\left(\frac{d\bar{\zeta}\wedge d\zeta}{\zeta -
    z}\right) + f\der{}{\bar{\zeta}}\left(\frac{1}{\zeta -
    z}\right)d\bar{\zeta}\wedge d\zeta
\]
We then use the fact that
\[
\der{}{\bar{\zeta}}\left(\frac{1}{\zeta - z}\right) = 0
\]
To conclude that
\[
\bar{\partial}\left(\frac{F(\zeta) d\zeta}{\zeta - z}\right) =
\der{F}{\bar{\zeta}}\left(\frac{d\bar{\zeta}\wedge d\zeta}{\zeta -
    z}\right)
\]
Substituting this into the integral formula we see that
\[
\int_{D_\delta(z)} (\partial + \bar{\partial})\frac{F(\zeta)
  d\zeta}{\zeta - z} = \int_{D_\delta(z)}
\der{F}{\bar{\zeta}}\left(\frac{d\bar{\zeta}\wedge d\zeta}{\zeta -
    z}\right)
\]
Now we will simplify the integral
\[
\int_{\partial D_\delta(z)} \frac{F(\zeta) d\zeta}{\zeta - z} =
\int_{\partial D_{\delta}(z)} \frac{F(\zeta) - f(z)}{\zeta - z}d\zeta
+ \int_{\partial D_\delta(z)} \frac{F(z)}{\zeta - z}d\zeta
\]
Now because $F$ is $C^1$ we can find a constant $M$ such that
$|F(\zeta) - F(z)| \leq M|\zeta - z|$ on $\partial D_{\delta}(z)$. This
gives the following estimate
\[
\left| \int_{\partial D_{\delta}(z)} \frac{F(\zeta) - f(z)}{\zeta -
    z}d\zeta\right| \leq M\int_{\partial
  D_{\delta}(z)}\left|\frac{\zeta -z}{\zeta -z}\right| |d\bar{\zeta}|
= 2\pi \delta M
\]
Letting $\delta \to 0$ forces the integral to 0 as well. Finally, we recall that
\[
\int_{\partial D_{\delta}(z)} \frac{F(z)}{\zeta - z}d\zeta = 2\pi
iF(z)
\]
This implies that
\[
\int_{\partial \Omega} \frac{F(\zeta)}{\zeta - z}d\zeta - 2\pi iF(z) =
\int_{\Omega_{(\delta,z)}}
\der{F}{\bar{\zeta}}\left(\frac{d\bar{\zeta}\wedge d\zeta}{\zeta -
    z}\right) + O(\delta)
\]
Letting $\delta \to 0$ and rearranging the terms yields
\[
F(z) = \frac{1}{2\pi i}\int_{\partial \Omega} \frac{F(\zeta)}{\zeta -
  z}d\zeta - \frac{1}{\pi}\int_\Omega \frac{(\partial
  F/\partial\bar{\zeta})(\zeta)}{\zeta - z}dm(\zeta)
\]

\exercise{7.9.7}

\end{document}