\documentclass{article}
\usepackage[tmargin=1in,bmargin=1in,lmargin=1.5in,rmargin=1.5in]{geometry}
\usepackage{amsfonts,amsmath,amssymb,amsthm}
\usepackage{mathrsfs}
\usepackage{ccfonts}
\usepackage{relsize,fancyhdr,parskip}
\usepackage{graphicx}

\usepackage{tikz} \usetikzlibrary{shapes, shapes.geometric,
  shapes.symbols, shapes.arrows, shapes.multipart, shapes.callouts,
  shapes.misc,decorations.markings,decorations.shapes}

\pagestyle{fancy}
\lhead{Ben Carriel}
\chead{Math 6120 Problem Set 5}
\rhead{\today}

\headheight 13.0pt
\parskip 7.2pt
\parindent 8pt

\newcommand{\tab}{\hspace*{2em}}
\newcommand{\tand}{\tab\text{and}\tab}

\DeclareMathOperator{\N}{\mathbb{N}}
\DeclareMathOperator{\Z}{\mathbb{Z}}
\DeclareMathOperator{\Q}{\mathbb{Q}}
\DeclareMathOperator{\R}{\mathbb{R}}
\DeclareMathOperator{\C}{\mathbb{C}}
\DeclareMathOperator{\D}{\mathbb{D}}
\DeclareMathOperator{\T}{\mathbb{T}}
\DeclareMathOperator{\capchi}{\raisebox{2pt}{$\mathlarger{\mathlarger{\chi}}$}}

\DeclareMathOperator{\divides}{\mathrel{|}}
\DeclareMathOperator{\suchthat}{\mathrel{:}}

\DeclareMathOperator{\lra}{\longrightarrow}
\DeclareMathOperator{\into}{\hookrightarrow}
\DeclareMathOperator{\onto}{\twoheadrightarrow}
\DeclareMathOperator{\bijection}{\leftrightarrow}
\DeclareMathOperator{\lap}{\bigtriangleup}

\newcommand{\problem}[1]{\noindent{\textbf{Problem #1}}\\}
\newcommand{\problempart}[1]{\noindent{\textbf{(#1)}}}
\newcommand{\exercise}[1]{\noindent{\textbf{Exercise #1:}}}

\newcommand{\der}[2]{\frac{\partial #1}{\partial #2}}
\newcommand{\norm}[1]{\|#1\|}
\newcommand{\diam}[1]{\text{diam}(#1)}
\newcommand{\seq}[2]{\{#1_{#2}\}_{#2 = 1}^\infty}

\newcommand{\conj}[1]{\overline{#1}}
\newcommand{\cis}[1]{\operatorname{cis}#1}
\newcommand{\res}{\operatorname{res}}

\newcommand{\real}{\mathrel{\text{Re}}}
\newcommand{\imag}{\mathrel{\text{Im}}}
\newtheorem*{thm}{\\ Theorem}
\newtheorem*{lem}{\\ Lemma}
\newtheorem*{claim}{\\ Claim}
\newtheorem*{defn}{\\ Definition}
\newtheorem*{prop}{\\ Proposition}

\begin{document}
\exercise{5.6.4}
\begin{enumerate}
\item[\textbf{(a)}] We are interested in
\[
F(z) = \prod_{n=1}^\infty (1 - e^{-2\pi nt}e^{2\pi iz})
\]
Following the hint, we consider the functions
\[
F_1(z) = \prod_{n=1}^N (1 - e^{-2\pi nt}e^{2\pi iz})
\]
and
\[
F_2(z) = \prod_{n=N+1}^\infty (1 - e^{-2\pi nt}e^{2\pi iz})
\]
Where $N$ is chosen such that $N \approx c|z|$ for a choice of $c$ we
will determine later. We then look at the behavior of $F_2$. We note
that
\[
\prod_{n=N+1}^\infty (1 - e^{-2\pi nt}e^{2\pi iz}) =
\prod_{n=N+1}^\infty e^{\log{(1 - e^{-2\pi nt}e^{2\pi iz})}} = e^{S_{N+1}}
\]
where
\[
S_{N+1} = \sum_{n=N+1}^\infty \log{(1 - e^{-2\pi nt}e^{2\pi iz})}
\]
Now we will choose $c$ such that $S_{N+1}$ will be bounded. Indeed,
consider $c = (1/t + \epsilon)$ for any $\epsilon > 0$. We then
observe that because of our choice of $c$ and for sufficiently large
$z$
\[
e^{-2\pi(N+1)t}e^{2\pi|z|} = e^{-2\pi((N+1)t - |z|)} \leq e^{-2\pi
  t(\epsilon |z| + 1)} \leq 1/2
\]
We then note that $|\log (1+x)| \leq 2|x|$ when $|x| \leq 1/2$ so
\[
|S_{N+1}| = \left|\sum_{N+1}^\infty \log{(1 - e^{-2\pi nt}e^{2\pi
      iz})}\right| \leq \sum_{N+1}^\infty |\log{(1 - e^{-2\pi nt}e^{2\pi
      iz})}|
\]
Applying the inequality from our choice of $c$ gives us
\[
\sum_{N+1}^\infty |\log{(1 - e^{-2\pi nt}e^{2\pi iz})}| \leq
\sum_{N+1}^\infty 2|e^{-2\pi nt}e^{2\pi iz}| \leq 2\sum_{N+1}^\infty
e^{-2\pi nt}e^{2\pi |z|}
\]
Changing indices yields
\[
2\sum_{N+1}^\infty e^{-2\pi nt}e^{2\pi |z|} = 2e^{-2\pi((N+1)t -
  |z|)}\sum_{m=0}^\infty e^{-2\pi mt}
\]
We note that the factor outside the sum is less than 1 and that the
series converges by the geometric series test. These two facts imply
that $|F_2(z)| \leq A$ for some constant $A$.

We now turn our attention for $F_1(z)$. Again, we know that for any
$n$
\[
|1 - e^{-2\pi nt}e^{2\pi iz}| \leq 1 + e^{2\pi|z|} \leq 2e^{2\pi |z|}
\]
As a result
\[
|F_1(z)| = \left|\prod_{n=1}^{N} (1 - e^{-2\pi nt}e^{2\pi iz})\right|
\leq \prod_{n=1}^N |1 - e^{-2\pi nt}e^{2\pi iz}| \leq \prod_{n=1}^N
2e^{2\pi |z|} = 2^Ne^{2\pi N|z|}
\]
Then because $N = O(|z|)$ we can modify the constants so that
\[
|F_1(z)| \leq e^{a|z|^2}
\]
We then combine the result from $F_1$ and $F_2$ to see that
\[
F(z) \leq Ae^{a|z|^2}
\]
And so $F$ is of order 2.
\item[\textbf{(b)}] We apply Proposition 5.3.1 to verify that $F$ is
  zero exactly when $z = -int + m$ for $n\geq 1$, $n,m \in
  \Z$. Indeed, by the proposition $F$ is zero precisely when one of
  its factors is zero. So we need solutions of the equation
  \[
  e^{-2\pi nt}e^{2\pi iz} = 1
  \]
  This happens when $nt - iz = 0$ so $z = -int + m$ for some integer $m$.

  Let $z_n$ be an enumeration of the zeros. To show that $\sum 1/|z_n|^2
  = \infty$ we note that by construction
  \[
  \sum_{n=1}^\infty \frac{1}{|z_n|^2} =
  \sum_{n=1}^\infty\sum_{m=-\infty}^\infty \frac{1}{|-int+m|^2} =
  \sum_{n=1}^\infty\sum_{m=-\infty}^{\infty} \frac{1}{m^2 + (nt)^2}
  \]
  Then recall the identity
  \[
  \pi\cot\pi a = \sum_{n=-\infty}^\infty \frac{a}{a^2 + n^2}
  \]
  Substituting this in our sum gives
  \[
  \sum_{n=1}^\infty\sum_{m=-\infty}^{\infty} \frac{1}{m^2 + (nt)^2} =
  \frac{1}{nt}\sum_{m=-\infty}^\infty \frac{nt}{(nt)^2 + m^2} =
  \sum_{n=1}^\infty \frac{\pi\coth \pi nt}{nt}
  \]
  The last sum diverges by the comparison test. Hence, $\sum 1/|z_n|^2
  = \infty$.

  To see that $\sum 1/|z_n|^{2+\epsilon}$ converges. We note that by
  part \textbf{(a)} $F$ has order 2 so for any $\epsilon > 0$ we
  satisfy the hypotheses for Theorem 5.2.1 which says precisely that
  \[
  \sum_{n=1}^\infty \frac{1}{|z_n|^{2+\epsilon}} < \infty
  \]
\end{enumerate}

\exercise{5.6.5}

We are looking at
\[
F_\alpha(z) = \int_{-\infty}^\infty e^{-|t|^\alpha}e^{2\pi izt}dt
\]
for $\alpha > 1$. We will take the hint and begin by showing that
\[
-\frac{|t|^\alpha}{2} + 2\pi |z||t| \leq c|z|^{\alpha / (\alpha-1)}
\]
First we consider the case that $|z| \leq
(1/4\pi)|t|^{\alpha-1}$. This implies that $2\pi |z||t| \leq
|t|^\alpha/2$ and so we have that
\[
-\frac{|t|^\alpha}{2} + 2\pi |z||t| \leq 0 \leq
c|z|^{\alpha/(\alpha-1)}
\]
Alternatively if $|z| > (1/4\pi)|t|^{\alpha-1}$ then we can set
\[
c = \frac{2\pi}{(4\pi)^{(\alpha-1)^{-1}}}
\]
In which case we have that $2\pi |z||t| \leq
c|z|^{\alpha/(\alpha-1)}$. This gives that
\[
-\frac{|t|^\alpha}{2} + 2\pi |z||t| \leq -\frac{|t|^\alpha}{2} +
c|z|^{\alpha/(\alpha-1)} \leq c|z|^{\alpha/(\alpha-1)}
\]
as desired.

We then apply the above inequality to see
\[
|F_\alpha(z)| \leq \int_{-\infty}^\infty
e^{-|t|^\alpha}e^{2\pi|z||t|}dt = \int_{-\infty}^\infty
e^{-|t|^\alpha/2}e^{2\pi|z||t|-|t|^\alpha/2}dt \leq
e^{c|z|^{\alpha/(\alpha-1)}}\int_{-\infty}^\infty e^{-|t|^\alpha/2}dt
\]
We then note that
\[
\int_{-\infty}^\infty e^{-|t|^\alpha/2}dt = \sqrt{2\pi}
\]
So that
\[
|F_\alpha(z)| \leq \sqrt{2\pi}e^{c|z|^{\alpha/(\alpha-1)}}
\]
So that $F$ is of growth order $\alpha/(\alpha-1)$ as desired.

\exercise{5.6.6}

We will evaluate the product formula for $\sin z$
\[
\frac{\sin \pi z}{\pi} = z \prod_{n=1}^\infty \left(1 - \frac{z^2}{n^2}\right)
\]
Changing variables to $w = \pi z$ we get
\[
\frac{\sin w}{w} = \prod_{n=1}^\infty\left(1 - \frac{w^2}{\pi^2n^2}\right)
\]
Following the hint we evaluate at $w = \pi/2$ to get
\[
\frac{2}{\pi} = \prod_{n=1}^\infty\left(1 - \frac{1}{4n^2}\right)
\]
Inverting both sides yields
\[
\frac{\pi}{2} = \prod_{n=1}^\infty\left(\frac{4n^2}{4n^2-1}\right) =
\prod_{n=1}^\infty \frac{2n\cdot 2n}{(2n-1)(2n+1)}
\]
as desired.

\exercise{5.6.8}

First let
\[
F(z) = \prod_{k=1}^\infty \cos (z/2^k)
\]
We will first address the issue of convergence. Consider the function
\[
G(z) = \prod_{n=1}^\infty (1 - \cos (z/2^n))
\]
We will show that $G$ converges, and then apply Proposition 5.3.1 to
get that $F$ must converge as well. If we look the formula
\[
\cos z = \sum_{n=0}^\infty \frac{(-1)^n}{(2n)!}z^{2n}
\]
Then when $|z| \approx 0$ we have that $|1-\cos z| \leq |z|^2$. As a
result, for sufficiently large $n$ we have that
\[
|1-\cos(z/2^n)| \leq (z/2^n)^2 = z^2/4^k
\]
Thus, $G$ converges because $\sum_{n > N} |z^2|/4^n$ converges
(geometric series test) and by Proposition 5.3.1 $F$ must converge as
well.

Now we follow the hint and repeatedly apply the identity
\[
\sin 2z = 2\sin z\cos z
\]
This will yield the formula
\begin{align*}
  \sin z &= 2\sin(z/2)\cos(z/2) \\
  &= 4\sin(z/4)\cos(z/2)\cos(z/4) \\
  &= 8\sin(z/8)\cos(z/2)\cos(z/4)\cos(z/8) \\
  &\vdots \\
  &= 2^n\sin(z/2^n)\prod_{k=1}^n \cos(z/2^k)
\end{align*}
Because $\sin |z| \approx |z|$ when $|z|$ is small, we see that
\[
\sin z = \lim_{n\to\infty} 2^n\sin(z/2^n)\prod_{k=1}^n \cos(z/2^k) = zF(z)
\]
Which precisely means that
\[
\frac{\sin z}{z} = \prod_{k=1}^\infty \cos(z/2^k)
\]

\exercise{5.6.9}

We first define
\[
F(z) = \prod_{k=0}^\infty (1 + z^{2^k})
\]
Adressing the issue of convergence, we note that $\sum_{k=0}^\infty
z^{2^k}$ converges in the disk $|z| < 1$ by comparison with the
geometric series. We apply Proposition 5.3.1 to conclude that $F$
converges as well.

We now need to show that $F(z) = 1/(1-z)$. We recall that
\[
\frac{1}{1-z} = \sum_{n=0}^\infty z^k
\]
for $|z| < 1$. We will proceed to show that the product formula for
$F(z)$ is equivalent to this summation formula. More specifically we
will show that
\[
\prod_{k=0}^n (1+z^{2^k}) = \sum_{\ell=0}^{2^{n+1}-1} z^\ell
\]
We proceed by induction. The base case $n=0$ is clear. For the
inductive step observe that
\[
\prod_{k=0}^{n+1} (1+z^{2^k}) = (1+z^{2^{n+1}})\prod_{k=0}^n
(1+z^{2^k}) = (1+z^{2^{n+1}})\left(\sum_{\ell=0}^{2^{n+1}-1} z^\ell\right)
\]
The last equality follows from the induction hypothesis. We then
distribute to see
\[
(1+z^{2^{n+1}})\left(\sum_{\ell=0}^{2^{n+1}-1} z^\ell\right) =
\left(\sum_{\ell=0}^{2^{n+1}-1} z^\ell\right) +
z^{2^{n+1}}\left(\sum_{\ell=0}^{2^{n+1}-1} z^\ell\right) =
\left(\sum_{\ell=0}^{2^{n+1}-1} z^\ell\right) +
\left(\sum_{\ell=2^{n+1}}^{2^{n+2}-1} z^\ell\right)
\]
We then combine the two sums two conclude that
\[
\prod_{k=0}^{n+1} (1+z^{2^k}) = \sum_{\ell=0}^{2^{n+2}-1} z^\ell
\]
and so equality holds for all $n$ by induction.

We then observe that
\[
F(z) = \lim_{n\to\infty} \prod_{k=0}^n (1+z^{2^k}) = \lim_{n\to\infty}
\sum_{k=0}^\infty z^k = \frac{1}{1-z}
\]

\exercise{5.6.15}

Let $f$ be a meromorphic function with poles at $\{z_n\}$, counted
with multiplicity. Then Theorem 5.4.1 guarantees the existence of an
entire function $g$ that vanishes precisely at the $\{z_n\}$ and
nowhere else. Then the product $h = fg$ is an entire function and we
can write $f = h/g$, so $f$ is the quotient of two entire functions.

Now let $\{a_n\}$ and $\{b_n\}$ be sequences that have no finite limit
point in the plane. Applying Theorem 5.4.1 to both of these sequences
yields two functions $f$ and $g$ with zeros precisely at the $\{a_n\}$
and $\{b_n\}$, respectively. Then the quotient $f/g$ has zeroes at the
$a_n$ and poles at the $b_n$.

\exercise{5.6.17}
\begin{enumerate}
\item[\textbf{(a)}] This part is the complex version Lagrange
  Interpolation, which I have seen before. Indeed, we have our two
  sets of complex numbers $\{a_1,\ldots,a_n\}$ and
  $\{b_1,\ldots,b_n\}$ with the $a_n$ all distinct. We will look to
  have some polynomial of the form
  \[
  P(z) = \sum_{k=1}^n b_kp_k(z)
  \]
  where $p_k(z)$ is a polynomial with the property that
  \[
  p_k(z) =
  \begin{cases}
    1 & z = a_k \\
    0 & z = a_\ell, \ell \neq j
  \end{cases}
  \]
  This will give $P$ the desired property that $P(a_k) = b_k$. So the
  construction of $P$ has been reduced to the construction of the $p_k$.

  We will then define a product formula for $p_k(z)$ via
  \[
  p_k(z) = \prod_{\substack{m \leq n \\ m \neq k}} \frac{z - a_m}{a_k - a_m}
  \]
  To see that $p_k$ is the desired function we note that for $j\neq k$
  \[
  p_k(a_j) = \prod_{\substack{m \leq n \\ m \neq k}} \frac{a_j -
    a_m}{a_k - a_m} = \left(\frac{a_j - a_j}{a_k -
      a_j}\right)\prod_{\substack{m \leq n \\ m \neq j,k}} \frac{z -
    a_m}{a_k - a_m} = 0
  \]
  Moreover,
  \[
  p_k(a_k) = \prod_{\substack{m \leq n \\ m \neq k}} \frac{a_k -
    a_m}{a_k - a_m} = 1
  \]
  as desired.
\item[\textbf{(b)}] We need to verify that the formula
  \[
  F(z) = \frac{b_0}{E'(a_0)}\frac{E(z)}{z}+\sum_{k=1}^\infty
  \frac{b_k}{E'(a_k)}\frac{E(z)}{z-a_k}\left\frac{z}{a_k}^{m_k}
  \]
  has the desired properties. Where $E(z)$ is a Weierstrass product
  for the $\{a_k\}$. As a result there is a zero with the correct
  multiplicity at each of the $a_k$. Moreover, a direct substitution
  shows that $F(a_k) = b_k$, given a suitably chosen value for the
  $m_k$. Let $F_R(z)$ be the restriction of $F$ to the disk of radius
  $R$ centered at the origin. We need to show that we can choose the
  $m_k$ suitably for each of the roots $a_k < R$. Indeed, we appeal to
  the discussion in Remark
\end{enumerate}

\end{document}