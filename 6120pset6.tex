\documentclass{article}
\usepackage[tmargin=1in,bmargin=1in,lmargin=1.5in,rmargin=1.5in]{geometry}
\usepackage{amsfonts,amsmath,amssymb,amsthm}
\usepackage{mathrsfs}
\usepackage{ccfonts}
\usepackage{relsize,fancyhdr,parskip}
\usepackage{graphicx}

\usepackage{tikz} \usetikzlibrary{shapes, shapes.geometric,
  shapes.symbols, shapes.arrows, shapes.multipart, shapes.callouts,
  shapes.misc,decorations.markings,decorations.shapes}

\pagestyle{fancy}
\lhead{Ben Carriel}
\chead{Math 6120 Problem Set 5}
\rhead{\today}

\headheight 13.0pt
\parskip 7.2pt
\parindent 8pt

\newcommand{\tab}{\hspace*{2em}}
\newcommand{\tand}{\tab\text{and}\tab}

\DeclareMathOperator{\N}{\mathbb{N}}
\DeclareMathOperator{\Z}{\mathbb{Z}}
\DeclareMathOperator{\Q}{\mathbb{Q}}
\DeclareMathOperator{\R}{\mathbb{R}}
\DeclareMathOperator{\C}{\mathbb{C}}
\DeclareMathOperator{\D}{\mathbb{D}}
\DeclareMathOperator{\T}{\mathbb{T}}
\DeclareMathOperator{\capchi}{\raisebox{2pt}{$\mathlarger{\mathlarger{\chi}}$}}

\DeclareMathOperator{\divides}{\mathrel{|}}
\DeclareMathOperator{\suchthat}{\mathrel{:}}

\DeclareMathOperator{\lra}{\longrightarrow}
\DeclareMathOperator{\into}{\hookrightarrow}
\DeclareMathOperator{\onto}{\twoheadrightarrow}
\DeclareMathOperator{\bijection}{\leftrightarrow}
\DeclareMathOperator{\lap}{\bigtriangleup}

\newcommand{\problem}[1]{\noindent{\textbf{Problem #1}}\\}
\newcommand{\problempart}[1]{\noindent{\textbf{(#1)}}}
\newcommand{\exercise}[1]{\noindent{\textbf{Exercise #1:}}}

\newcommand{\der}[2]{\frac{\partial #1}{\partial #2}}
\newcommand{\norm}[1]{\|#1\|}
\newcommand{\diam}[1]{\text{diam}(#1)}
\newcommand{\seq}[2]{\{#1_{#2}\}_{#2 = 1}^\infty}

\newcommand{\conj}[1]{\overline{#1}}
\newcommand{\cis}[1]{\operatorname{cis}#1}
\newcommand{\res}{\operatorname{res}}

\newcommand{\real}{\mathrel{\text{Re}}}
\newcommand{\imag}{\mathrel{\text{Im}}}
\newtheorem*{thm}{\\ Theorem}
\newtheorem*{lem}{\\ Lemma}
\newtheorem*{claim}{\\ Claim}
\newtheorem*{defn}{\\ Definition}
\newtheorem*{prop}{\\ Proposition}

\begin{document}
\exercise{6.3.5}

We will use the fact that
\[
\Gamma(s)\Gamma(1-s) = \frac{\pi}{\sin\pi s}
\]
And the following
\begin{claim}
  $\overline{\Gamma(s)} = \Gamma(\overline{s})$
\end{claim}
\begin{proof}
  This is clear from the product formula for $\Gamma(s)$. By Theorem
  6.1.7 we see
  \[
  \frac{1}{\Gamma(s)} = e^{\gamma s}s\prod_{n=1}^\infty \left(1 +
    \frac{s}{n}\right)e^{-s/n}
  \]
  Consequently,
  \[
  \Gamma(z) = \frac{e^{-\gamma s}}{s}\prod_{n=1}^\infty \left(1 +
    \frac{s}{n}\right)^{-1}e^{s}{n}
  \]
  Then we note that complex conjugates are preserved under products
  and take the limit of the comjugates of partial products to get that
  \[
  \overline{\Gamma(s)} = \Gamma(\overline{s})
  \]
\end{proof}
We can then compute
\[
|\Gamma(1/2+it)|^2 = \Gamma(1/2+it)Gamma(1/2-it) =
\frac{\pi}{\sin\pi(1/2+it)}
\]
Then note that $\sin(\theta+\pi/2) = \cos(\theta)$ to see that
\[
\frac{\pi}{\sin\pi(1/2+it)} = \frac{\pi}{\cos\pi it} =
\frac{\pi}{\cosh \pi t} = \frac{2\pi}{e^{\pi t} + e^{-\pi t}}
\]
Then
\[
|\Gamma(1/2+it)|^2 = \frac{2\pi}{e^{\pi t} + e^{-\pi t}}
\]
Taking square roots yields
\[
|\Gamma(1/2+it)| = \sqrt{\frac{2\pi}{e^{\pi t} + e^{-\pi t}}}
\]

\exercise{6.3.7}
\begin{enumerate}
\item[\textbf{(a)}] We first observe that
\begin{align*}
\Gamma(\alpha)\Gamma(\beta) &= \left(\int_0^\infty
  e^{-t}t^{\alpha-1}dt\right)\left(\int_0^\infty
  e^{-s}s^{\beta-1}ds\right) \\
&= \int_0^\infty\int_0^\infty t^{\alpha-1}s^{\beta-1}e^{-t-s}dtds
\end{align*}
Following the hint, we change variables to $s = ur, t = u(1-r)$ and
compute
\begin{align*}
  \int_0^\infty\int_0^\infty t^{\alpha-1}s^{\beta-1}e^{-t-s}dtds &=
  \int_0^\infty\int_0^1(u(1-r))^{\alpha-1}(ur)^{\beta-1}e^{-u}udrdu \\
  &= \left(\int_0^\infty
    u^{\alpha+\beta-1}e^{-u}du\right)\left(\int_0^1
    r^{\beta-1}(1-r)^{\alpha-1}dr\right) \\
  &= \Gamma(\alpha+\beta)B(\alpha,\beta)
\end{align*}
Dividing on both sides gives
\[
B(\alpha,\beta) = \frac{\Gamma(\alpha)\Gamma(\beta)}{\Gamma(\alpha+\beta)}
\]
\item[\textbf{(b)}] This also follows by a change of
  variable. Inspection of the formula suggests that the thing to do set
  \[
  t = \frac{1}{1+u}\tab 1-t = \frac{u}{1+u}
  \]
  Consequently, we have that
  \[
  dt = \frac{-1}{(1+u)^2}du
  \]
  Hence,
  \begin{align*}
    B(\alpha,\beta) = \int_0^1 (2-t)^{\alpha-1}t^{\beta-1}dt &=
    -\int_0^\infty
    \left(\frac{u}{1+u}\right)^{\alpha-1}\left(\frac{1}{1+u}\right)^{\beta-1}
    \left(\frac{-1}{(1+u)^2}\right)du\\
    &= \int_0^\infty \frac{u^{\alpha-1}}{(1+u)^{\alpha+\beta}}du
  \end{align*}
  as desired.
\end{enumerate}

\exercise{6.3.11}

We begin with the function
\[
f(z) = e^{az}e^{-e^z}
\]
We then recall the definition
\[
\hat{f}(\xi) = \int_{-\infty}^\infty f(x)e^{-2\pi ix\xi}dx
\]
and then compute
\[
\int_{-\infty}^\infty f(x)e^{-2\pi ix\xi}dx = \int_{-\infty}^\infty
e^{ax}e^{-e^x}e^{-2\pi ix\xi}dx = \int_{-\infty}^\infty e^{x(a-2\pi
i\xi)}e^{-e^x}dx
\]
Then make the change of variable $t = e^x$ so that
\[
\int_{-\infty}^\infty e^{x(a-2\pi i\xi)}e^{-e^x}dx =
\int_{0}^\infty t^{(a-2\pi i\xi)}e^{-t}\frac{dt}{t} =
\int_{0}^\infty t^{(a-2\pi i\xi -1)}e^{-t}dt
\]
Then observe that
\[
\Gamma(a - 2\pi i\xi) = \int_{0}^\infty t^{(a-2\pi i\xi -1)}e^{-t}dt
\]
So that
\[
\hat{f}(\xi) = \Gamma(a - 2\pi i\xi)
\]

\exercise{6.3.13}

We recall from the textbook that
\[
\frac{1}{\Gamma(s)} = e^{\gamma s}s\prod_{n=1}^\infty \left(1 +
  \frac{s}{n}\right)e^{-s/n}
\]
Taking the logarithm yields
\[
-\log\Gamma(s) = \log s + \gamma s + \sum_{n=1}^\infty\log\left(1 +
  \frac{s}{n}\right) - \frac{s}{n}
\]
We then note that we can differentiate the sum term by term because
the convergence is uniform on compact sets that do not contain a
singularity. We then compute for $s > 0$
\[
\frac{d^2}{ds^2}(-\log\Gamma(s)) = \frac{-1}{s^2} + \sum_{n=1}^\infty
\frac{-1}{(n+s)^2} = \sum_{n=0}^\infty \frac{-1}{(n+s)^2}
\]
The right hand side is an analytic function whenever $s$ is not a
negative integer because the sum converges uniformly in this
region. If we consider the left hand side to be $(\Gamma'/\Gamma)'$
then we observe that both functions are analytic, and $\Gamma$ is
non-vanishing and the formula must hold by analytic continuation.

\exercise{6.3.15}

We suppose that $\real(s) > 1$ and need to show that
\[
\zeta(s) = \frac{1}{\Gamma(s)}\int_0^\infty \frac{x^{s-1}}{e^x-1}dx
\]
Following the hint, we note that
\[
\frac{1}{e^x-1} = \sum_{n=1}^\infty e^{-nx}
\]
We then see that
\[
\int_0^\infty \frac{x^{s-1}}{e^x-1}dx = \int_0^\infty
x^{s-1}\sum_{n=1}^\infty e^{-nx}dx = \int_0^\infty \sum_{n=1}^\infty
x^{s-1}e^{-nx}dx
\]
We then note that on any interval $(\epsilon,1/\epsilon)$ the
convergence of the sum is uniform and we can exchange the sum and the
integral. So we then compute
\begin{align*}
  \int_\epsilon^{1/\epsilon} \sum_{n=1}^\infty x^{s-1}e^{-nx}dx &=
  \sum_{n=1}^\infty \int_\epsilon^{1/\epsilon}x^{s-1}e^{-nx}dx \\
  &= \sum_{n=1}^\infty \int_{n\epsilon}^{n/\epsilon} \left(\frac{t}{n}
  \right)^{s-1} e^{-t}dt
\end{align*}
Factoring gives
\[
\sum_{n=1}^\infty \int_{n\epsilon}^{n/\epsilon} \left(\frac{t}{n}
\right)^{s-1} e^{-t}dt = \sum_{n=1}^\infty
\frac{1}{n^s}\int_{n\epsilon}^{n/\epsilon} t^{s-1}e^{-t}dt
\]
We then note that because this holds for every $\epsilon > 0$ we can
apply dominated convergence to let $\epsilon \to 0$ and yield
\[
\int_0^\infty \frac{x^{s-1}}{e^x-1}dx = \sum_{n=1}^\infty
\frac{1}{n^s}\int_{0}^{\infty} t^{s-1}e^{-t}dt = \zeta(s)\Gamma(s)
\]
Dividing both sides by $\Gamma(s)$ gives
\[
\zeta(s) = \frac{1}{\Gamma(s)}\int_0^\infty \frac{x^{s-1}}{e^x-1}dx
\]

\exercise{6.3.16}

By the previous exercise we have that
\[
\zeta(s) = \frac{1}{\Gamma(s)}\int_0^\infty \frac{x^{s-1}}{e^x-1}dx
\]
We then use linearity to see that
\[
\zeta(s) = \frac{1}{\Gamma(s)}\int_0^1 \frac{x^{s-1}}{e^x-1}dx +
\frac{1}{\Gamma(s)}\int_1^\infty \frac{x^{s-1}}{e^x-1}dx
\]
We see that the second term is an entire function becuase $1/\Gamma$
is entire. For the first term we must consider the behavior of
\[
\int_0^1 \frac{x^{s-1}}{e^x-1}dx = \int_0^1 x^{s-2}\frac{x}{e^x-1}dx
\]
We then recall the definition of the Bernoulli numbers
\[
\frac{x}{e^x-1} = \sum_{m=0}^\infty \frac{B_m}{m!}x^m
\]
This gives
\[
\int_0^1 x^{s-2}\frac{x}{e^x-1}dx = \int_0^1 x^{s-2}\sum_{m=0}^\infty
\frac{B_m}{m!}x^mdx
\]
We then note that because the convergence is uniform (it is the Taylor
series) in the disk $|z| < 2\pi$ we can exchange the sum and the
integral to get
\[
\int_0^1 x^{s-2}\sum_{m=0}^\infty \frac{B_m}{m!}x^mdx =
\sum_{m=0}^\infty\frac{B_m}{m!}\int_0^1x^{m+s-2}dx = \sum_{m=0}^\infty
\frac{B_m}{m!(m+s-1)}
\]
Rewriting this gives
\[
\int_0^1 \frac{x^{s-1}}{e^x-1}dx = \sum_{m=0}^\infty
\frac{B_m}{m!(m+s-1)}
\]
Hence, the function is clearly analytic except possibly when $s =
1$. We then observe that we have an analytic continuation
\[
\frac{\Gamma(m+s-1)}{\Gamma(m+s)} = \frac{1}{m+s-1}
\]
When $s = 1$ we have a simple pole at $s=1$ and nowhere else. Thus, we
have defined a function that is continuable, using the above
extension, to the whole plane with a simple pole at $s=1$.
\end{document}