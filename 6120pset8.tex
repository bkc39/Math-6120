\documentclass[11pt]{article}
\usepackage[tmargin=1in,bmargin=1in,lmargin=1.5in,rmargin=1.5in]{geometry}
\usepackage{amsfonts,amsmath,amssymb,amsthm}
\usepackage{mathrsfs}
\usepackage{ccfonts}
\usepackage{relsize,fancyhdr,parskip}
\usepackage{graphicx}

\usepackage{tikz} \usetikzlibrary{shapes, shapes.geometric,
  shapes.symbols, shapes.arrows, shapes.multipart, shapes.callouts,
  shapes.misc,decorations.markings,decorations.shapes}

\pagestyle{fancy}
\lhead{Ben Carriel}
\chead{Math 6120 Problem Set 5}
\rhead{\today}

\headheight 13.0pt
\parskip 7.2pt
\parindent 8pt

\newcommand{\tab}{\hspace*{2em}}
\newcommand{\tand}{\tab\text{and}\tab}

\DeclareMathOperator{\N}{\mathbb{N}}
\DeclareMathOperator{\Z}{\mathbb{Z}}
\DeclareMathOperator{\Q}{\mathbb{Q}}
\DeclareMathOperator{\R}{\mathbb{R}}
\DeclareMathOperator{\C}{\mathbb{C}}
\DeclareMathOperator{\Ha}{\mathbb{H}}
\DeclareMathOperator{\D}{\mathbb{D}}
\DeclareMathOperator{\T}{\mathbb{T}}
\DeclareMathOperator{\capchi}{\raisebox{2pt}{$\mathlarger{\mathlarger{\chi}}$}}

\DeclareMathOperator{\divides}{\mathrel{|}}
\DeclareMathOperator{\suchthat}{\mathrel{:}}

\DeclareMathOperator{\lra}{\longrightarrow}
\DeclareMathOperator{\into}{\hookrightarrow}
\DeclareMathOperator{\onto}{\twoheadrightarrow}
\DeclareMathOperator{\bijection}{\leftrightarrow}
\DeclareMathOperator{\lap}{\bigtriangleup}

\newcommand{\problem}[1]{\noindent{\textbf{Problem #1}}\\}
\newcommand{\problempart}[1]{\noindent{\textbf{(#1)}}}
\newcommand{\exercise}[1]{\noindent{\textbf{Exercise #1:}}}

\newcommand{\der}[2]{\frac{\partial #1}{\partial #2}}
\newcommand{\norm}[1]{\|#1\|}
\newcommand{\lpnorm}[2]{\|#1\|_{L^{#2}}}
\newcommand{\diam}[1]{\text{diam}(#1)}
\newcommand{\seq}[2]{\{#1_{#2}\}_{#2 = 1}^\infty}

\newcommand{\conj}[1]{\overline{#1}}
\newcommand{\cis}[1]{\operatorname{cis}#1}
\newcommand{\res}{\operatorname{res}}

\newcommand{\real}{\mathrel{\text{Re}}}
\newcommand{\imag}{\mathrel{\text{Im}}}
\newtheorem*{thm}{\\ Theorem}
\newtheorem*{lem}{\\ Lemma}
\newtheorem*{claim}{\\ Claim}
\newtheorem*{defn}{\\ Definition}
\newtheorem*{prop}{\\ Proposition}

\begin{document}
\exercise{2.7.3}

Consider inequalities of the form
\[
\lpnorm{\hat{f}}{q} \leq A\lpnorm{f}{p}
\]
where $f$ is a simple function. We see in Corollary 2.2.6 that if $p,q$
are conjugate then such an inequality is possible. We need to show
that this is necessary. Indeed, if we look at the family of functions
$f_r(x) = f(rx)$ for $r > 0$. Then we compute
\[
\hat{f}_r(\xi) = \int f_r(x)e^{-2\pi ix\cdot \xi}dx = r^{-d}\int
f(u)e^{-2\pi iu\cdot \xi/r}du = r^{-d}\hat{f}_r(\xi/r)
\]
We then note that because $f$ is simple, it is in both $L^p$ and $L^q$
for all $p$ and $q$. This implies that the inequality above satisfies
\[
\lpnorm{\hat{f}}{q} \leq Ar^d\lpnorm{f}{p}
\]
due to translation invariance of the integral. We know by the
Hausdorff-Young inequality that the Fourier transform of an $L^p$
function is in $L^q$ if $q$ is conjugate to $p$. Letting $r\to\infty$
the right hand is unbounded we have a contradiction unless $p,q$ are
conjugate.

\exercise{2.7.4}

We are considering estimates of the type
\[
\int_{|\xi| \leq 1} |\hat{f}(\xi)|d\xi \leq A\lpnorm{f}{p}
\]
Given $f \in L^p$, consider the function
\[
f^s(x) = s^{-d/2}e^{-\pi|x|^2/s}
\]
where $s = \sigma + it$, $\sigma > 0$. We then compute the Fourier
transform $\hat{f^s}(\xi)$ via
\begin{align*}
  \hat{f^s}(\xi) &= \int_{\R^d} s^{-d/2}e^{-\pi|x|^2/s}e^{-2\pi ix\cdot \xi}dx
\end{align*}
Then note that because $|x|^2 = x \cdot x$ the inner products in the
exponents above allow the integral to admit a factorization into
single variable integrals yielding
\begin{align*}
  \hat{f^s}(\xi) &= \prod_{j=1}^ds^{-d/2}\int_{-\infty}^\infty e^{-\pi
    x_j^2/s}e^{-2\pi ix_j\xi_j}dx_j \\
  &= \prod_{j=1}^ds^{-d/2}\int_{-\infty}^\infty
  e^{-\frac{\pi}{s}(x_j-is\xi_j)^2}e^{-\pi s\xi_j^2}dx_j \\
  &= \prod_{j=1}^ds^{-d/2}e^{-\pi s\xi_j^2}\int_{-\infty}^\infty
  e^{-\frac{\pi}{s}(x_j-is\xi_j)^2}dx_j
\end{align*}
Where the third equality above follows by completing the square. We
can then simplify each term using contour integration using a
reduction to the Gaussian integral
\begin{align*}
  \int_{-\infty}^\infty e^{-\frac{\pi}{s}(x_j-is\xi_j)^2}dx_j =
  \int_{-\infty-is\xi_j}^{\infty-is\xi_j} e^{-\pi u_j^2/s}du_j =
  s^{1/2}
\end{align*}
Pluggin this into the product formula above yields
\[
\hat{f^s} = \prod_{j=1}^ds^{-d/2}e^{-\pi
  s\xi_j^2}\int_{-\infty}^\infty e^{-\frac{\pi}{s}(x_j-is\xi_j)^2}dx_j
= e^{-\pi s|\xi|^2}s^{-d/2}\prod_{j=1}^d s^{1/2} = e^{-\pi s|\xi|^2}
\]
Now we restrict our attention to the line $\sigma = 1$ so that $s = 1
+ it$. We then compute
\begin{align*}
  \lpnorm{f^s}{p}^p &= \int \left|s^{-d/2}e^{-\pi|x|^2/s}\right|^pdx \\
  &\leq |s^{-d/2}|^p\int \prod_{j=1}^d
  \left| e^{-\pi x_j^2/s}\right|^pdx_1\ldots dx_d \\
  &\leq |s^{-d/2}|\prod_{j=1}^\infty \int \left|e^{-\pi x_j^2/s}\right|^pdx_j \\
\end{align*}
Now we use the fact that $s = 1 + it$. Then we see that
\[
1/(1+it) = \frac{1}{t^2+1} - \frac{it}{t^2+1}
\]
So the exponent in the integral resolves to
\[
(-\pi ix_j^2)\left(\frac{1}{t^2+1} - \frac{it}{t^2+1}\right) =
\left(\frac{-\pi tx_j^2}{t^2+1} - \frac{\pi i x_j^2}{t^2+1}\right)
\]
Now we use the fact that
\[
\int e^{-\pi ax^2}dx = \frac{1}{\sqrt{a}}
\]
to continue the above computation as follows
\begin{align*}
  \int \left|e^{-\pi x_j^2/s}\right|^pdx_j &= \int \left|\real e^{-\pi
      x_j^2/s}\right|^pdx_j \\
  &= \int e^{-\pi (pt/t^2+1)x_j^2}dx_j \\
  &= \left(\frac{pt}{t^2+1}\right)^{-1/2}\\
  &= O(t^{1/2})
\end{align*}
Plugging this into the integral formula above gives
\[
\lpnorm{f^s}{p}^p \leq |(1+it)^{-d/2}|^pO(t^{1/2}) =
O(t^{-pd/2})O(t^{d/2}) = O(t^{pd - d/2})
\]
Then taking $p$-th roots gives $\lpnorm{f^s}{p} \leq ct^{d(1/p -
  1/2)}$. We then let $t \to \infty$ and observe that if $1/p < 1/2$
then the right hand side of the inequality
\[
\int_{|\xi| \leq 1} |\hat{f}(\xi)|d\xi \leq A\lpnorm{f}{p}
\]
will go to zero, but the left hand side will remain positive, a
contradiction. Hence, $1/p \geq 1/2$ so we must have that $p \leq 2$.

\exercise{2.7.5}

Let $S$ be the strip $0 < \real (z) < 1$. Let $\psi(z) = e^i\pi z$ and
let $\Phi(z) = e^{-i\psi(z)}$. First we will check that $\Phi$ is
continuous on the closure of $S$. Indeed, we note that $\varphi (z)=
e^{iz}$ is clearly continuous on the closed upper half-plane. So we
only need to verify that $\psi: S \to \mathbb{H}$ is continuous on the
closure of $S$. Suppose that $\alpha \in \partial S$, then because
$\psi$ is a conformal map, $\psi(z)$ must be on the boundary of
$\mathbb{H}$. Suppose that $\psi$ is not continuous, then we can find
two distinct sequences $\{z_i\}_{i \in \N}, \{w_i\}_{i \in \N}$
converging to $\alpha$ such that $\lim_{i \to \infty} \psi(z_i) \neq
\lim_{i \to \infty} \psi(w_i)$. These limits must exist in
$\mathbb{H}$ because the $z_i,w_i$ are bounded and convergent. If we
set $z,w$ as the respective limits then we know that their are open
sets $U_z,U_w$ such that $d(U_z,U_w) > \delta$ for some $\delta >
0$. But then the continuity of $\psi$ on the interior of $S$
guarantees that if $d(w_i,z_j) < \delta_0$ then $d(\psi(w_i),
\psi(z_j)) \leq \delta$. Because infinitely many of the $z_i,w_i$ are
contained in a $\delta_0$-neighborhood of $\alpha$ we have a
contradiction. Thus, $\psi$ is continuous on the boundary.

Now we need to see that $|\Phi(z)| = 1$ on the boundary of $S$. This
follows from the fact that angles are preserved under conformal maps,
we see that $\psi$ maps the boundary of $S$ to the boundary of
$\mathbb{H}$. This is $\R$ lying in $\C$ which is taken onto the
circle $S^1$ under the map $e^{iz}$, so that $|\Phi(z) = 1$. Finally,
to see that $Phi$ is unbounded on the interior of $S$ we note that it
is non-constant and holomorphic and therefore cannot be bounded.

\exercise{2.7.7}

We have that $f$ is bounded with compact support and therefore must be
in $L^2(\R)$. Then we must have that
\[
H(f)(x) = \lim_{\epsilon \to 0} \frac{1}{\pi}\int_{|t| \geq \epsilon}
f(x-t) \frac{dt}{t}
\]
Moreover, we see that
\[
\int_{-\infty}^\infty H(f)(x)dx = \int_{-\infty}^\infty
\left(\lim_{\epsilon\to 0} \frac{1}{\pi}\int_{|t| \geq \epsilon} f(x-t)
\frac{dt}{t}\right)dx = \lim_{\epsilon\to 0}\int_{-\infty}^\infty
\frac{1}{\pi}\int_{|t| \geq \epsilon} \frac{f(t)}{x-t}dt
\]
Note that the change of variable is ok because $f$ is bounded with
compact support. Then we integrate by parts to see that if $a = \int
fdx$ then $Hf(x) = a/\pi x + O(1/x^2)$. This clearly shows that $H(f)$
cannot be $L^1$ unless $a = 0$, which precisely says that $\int fdx =
0$. 

\exercise{2.7.10}
\begin{enumerate}
\item[(a)] Suppose $1 \leq p \leq \infty$ and recall that
  \[
  (f \ast \mathcal{P}_y) = \int_{-\infty}^\infty \hat{f}(\xi)e^{2\pi i
    x\xi}e^{-2\pi y|\xi|}d\xi
  \]
  Then for $f \in L^p(\R)$ we need to compute
  \[
  \lpnorm{f\ast \mathcal{P}_y}{p}^p =
  \int_{-\infty}^\infty\left|\int_{-\infty}^\infty \hat{f}(\xi)e^{2\pi
      i x\xi}e^{-2\pi y|\xi|}d\xi\right|^pdx
  \]
  Indeed, for the inner integral we observe that
  \[
  \left|\int_{-\infty}^\infty \hat{f}(\xi)e^{2\pi i x\xi}e^{-2\pi
      y|\xi|}d\xi\right|^p = \left|\int_{-\infty}^\infty
    \hat{f}(\xi)e^{2\pi i(x+i\mathrel{\text{sign}}(\xi)y)\xi}d\xi \right|^p
  \]
  Moving the absolute value inside the integral and applying the
  Fourier inversion formula to the aboslute value gives
  \[
  \left|\int_{-\infty}^\infty \hat{f}(\xi)e^{2\pi
      i(x+i\mathrel{\text{sign}}(\xi)y)\xi}d\xi \right|^p \leq
  \left(\int_{-\infty}^\infty \left|\hat{f}(\xi)e^{2\pi
      ix\xi}\right|d\xi\right)^p = |f(x)|^p
  \]
  Note that the imaginary part was eliminated when passing to the
  absolute value above. Substitution into the formula for the norm
  gives
  \[
  \lpnorm{f \ast \mathcal{P}_y}{p}^p \leq \int_{-\infty}^\infty
  |f(x)|^pdx = \lpnorm{f}{p}^p
  \]
  Taking $p^{\text{th}}$ roots gives the result.
\item[(b)] We first recall that
  \[
  \int_{-\infty}^\infty \mathcal{P}_y(x)dx = 1
  \]
  for each $y > 0$. As a result of this we see that
  \[
  (f \ast \mathcal{P}_y)(x) - f(x) = \int_{-\infty}^\infty [f(x-t) -
  f(x)]\mathcal{P}_y(t)dt
  \]
  Consequently,
  \[
  |(f \ast \mathcal{P}_y)(x) - f(x)| \leq \int_{-\infty}^\infty
  |f(x-t) - f(x)|^p\mathcal{P}_y(t)dt
  \]
  We then set $f_t(x) = f(x-t)$ integrate integrate with respect to $t$ and apply Fubini's
  theorem to get
  \[
  \lpnorm{f \ast \mathcal{P}_y)(x) - f(x)}{p}^p \leq
  \int_{-\infty}^\infty \lpnorm{f_t - f}{p}^p\mathcal{P}_y(t)dt
  \]
  The function $g(t) = \lpnorm{f_t - f}{p}^p$ is bounded and
  continuous because $t \mapsto f_t$ is uniformly continuous from $\R
  \to L^p(\R)$ and the fact that $g(0) = 0$. Thus, the right hand side
  of the above goes to zero as $y \to 0$.
\end{enumerate}
\end{document}