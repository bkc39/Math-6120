\documentclass{article}
\usepackage[tmargin=1in,bmargin=1in,lmargin=1.5in,rmargin=1.5in]{geometry}
\usepackage{amsfonts,amsmath,amssymb,amsthm}
\usepackage{mathrsfs}
\usepackage{ccfonts}
\usepackage{relsize,fancyhdr,parskip}
\usepackage{graphicx}

\usepackage{tikz} \usetikzlibrary{shapes, shapes.geometric,
  shapes.symbols, shapes.arrows, shapes.multipart, shapes.callouts,
  shapes.misc,decorations.markings,decorations.shapes}

\pagestyle{fancy}
\lhead{Ben Carriel}
\chead{Math 6120 Problem Set 9}
\rhead{\today}

\headheight 13.0pt
\parskip 7.2pt
\parindent 8pt

\newcommand{\tab}{\hspace*{2em}}
\newcommand{\tand}{\tab\text{and}\tab}

\DeclareMathOperator{\N}{\mathbb{N}}
\DeclareMathOperator{\Z}{\mathbb{Z}}
\DeclareMathOperator{\Q}{\mathbb{Q}}
\DeclareMathOperator{\R}{\mathbb{R}}
\DeclareMathOperator{\C}{\mathbb{C}}
\DeclareMathOperator{\Ha}{\mathbb{H}}
\DeclareMathOperator{\D}{\mathbb{D}}
\DeclareMathOperator{\T}{\mathbb{T}}
\DeclareMathOperator{\capchi}{\raisebox{2pt}{$\mathlarger{\mathlarger{\chi}}$}}

\DeclareMathOperator{\divides}{\mathrel{|}}
\DeclareMathOperator{\suchthat}{\mathrel{:}}

\DeclareMathOperator{\lra}{\longrightarrow}
\DeclareMathOperator{\into}{\hookrightarrow}
\DeclareMathOperator{\onto}{\twoheadrightarrow}
\DeclareMathOperator{\bijection}{\leftrightarrow}
\DeclareMathOperator{\lap}{\bigtriangleup}

\newcommand{\problem}[1]{\noindent{\textbf{Problem #1}}\\}
\newcommand{\problempart}[1]{\noindent{\textbf{(#1)}}}
\newcommand{\exercise}[1]{\noindent{\textbf{Exercise #1:}}}

\newcommand{\der}[2]{\frac{\partial #1}{\partial #2}}
\newcommand{\norm}[1]{\|#1\|}
\newcommand{\lpnorm}[2]{\|#1\|_{L^{#2}}}
\newcommand{\diam}[1]{\text{diam}(#1)}
\newcommand{\seq}[2]{\{#1_{#2}\}_{#2 = 1}^\infty}

\newcommand{\conj}[1]{\overline{#1}}
\newcommand{\cis}[1]{\operatorname{cis}#1}
\newcommand{\res}{\operatorname{res}}

\newcommand{\real}{\mathrel{\text{Re}}}
\newcommand{\imag}{\mathrel{\text{Im}}}
\newtheorem*{thm}{\\ Theorem}
\newtheorem*{lem}{\\ Lemma}
\newtheorem*{claim}{\\ Claim}
\newtheorem*{defn}{\\ Definition}
\newtheorem*{prop}{\\ Proposition}

\begin{document}
\exercise{3.4.3}

Suppose that $f$ is a bounded Lipschitz function with Lipschitz
constant $C$. Then $f \in L^\infty$. To reason about the first order
derivatives $\der{}{x_j}f$, we convolve with an approximate identity
as in Proposition 3.1.2 and define the sequence $f_n = f \ast \phi_n$,
where $\psi_n$ is the approximate identity. Then we see that
$\der{}{x_j}f_n = \der{}{x_j}f \ast \psi_n \to \der{}{x_j}f$ uniformly
with $n$. Then we compute the partial derivatives via the difference
quotient yielding
\begin{align*}
  \int \frac{f(x + y + \epsilon e_j) - f(x+y)}{\epsilon}\psi_n(y)dy
  &\leq C\int \psi_n(y)dy
\end{align*}
This goes to $C$ uniformly in $n$. So the partial derivatives are
uniformly bounded by the Lipshitz constant $C$.

\exercise{3.4.4}
\begin{enumerate}
\item[\textbf{(a)}] Because $F$ is a distribution on $\Omega$, we can
  apply Proposition 3.1.2 to get a sequence of $C^\infty$ functions
  $f_n$ such taht $f_n \to F$ in the sense of distributions. Now we
  use the fact that $C_{Com}^\infty \subset C^\infty$ is a dense
  containment to get a sequence of functions $f_{n,m} \to f_n$ as
  $m\to\infty$, where the convergence takes place in
  $C^\infty$. Because this convergence is in the sense of functions,
  we can conclude that $f_{n,m} \to f_n$ in the sense of distributions
  as well. Finally, we note that in the sense of distributions
  \[
  |f_{n.n}(\varphi) - F(\varphi)| = |(f_{n.n}(\varphi) -
  f_n(\varphi))+ (f_n(\varphi - F(\varphi))| \leq |(f_{n.n}(\varphi) -
  f_n(\varphi))| + |(f_n(\varphi - F(\varphi))|
  \]
  Letting $n \to \infty$ gives the result.
\item[\textbf{(b)}] Now let $C$ be the support of $F$. We need to
  restrict the approximating functions $f_n$ to an
  $\epsilon$-neighborhood of $C$. For $0 \leq \delta \leq \epsilon$
  let $C_\delta$ be all the points within $\delta$ of $C$/ Fix a
  $\delta_0 < \epsilon/2$ and choose a function $\varphi$ supported in
  the unit ball such that $\int \varphi = 1$. Then we consider the
  family of scaling functions
  \[
  \varphi_R(x) = R^{-d}\varphi(x/R)
  \]
  which are supported in the balls of radius $R$ about the origin. We
  then consider the convolutions
  \[
  \varphi_{\delta_0} \ast \capchi_{C_{\delta_0}} = \int_{\R^d}
  \varphi_{\delta_0}(y)\capchi_{C_{\delta_0}}(x-y)dy = \int_{|y| \leq
    \delta_0}\varphi_{\delta_0}(y)\capchi_{C_{\delta_0}}(x-y)dy
  \]
  because $\varphi_{\delta_0}$ is supported on the ball of radius
  $\delta_0$. Now if $x \in C$ and $|y| \leq \delta_0$ then
  $\capchi_{C_{\delta_0}}(x-y) = 1$, but if $x \not\in C_{\epsilon}$
  and $|y| \leq \delta_0$ then $x-y \not\in C_{\epsilon}$. Thus, we
  see that
  \[
  (\varphi_{\delta_0} \ast \capchi_{C_{\delta_0}})(x) =
  \begin{cases}
    1 & x \in C \\
    0 & \text{outside }C_{\epsilon}
  \end{cases}
  \]
  Now we simply note that $\varphi_{\delta_0} \ast
  \capchi_{C_{\delta_0}} \in \C_{Com}^\infty$ by construction and we
  then simply take a sequence $f_n \to F$ with the $f_n \in
  C^{\infty}$ and define the new sequence $(\varphi_{\delta_0} \ast
  \capchi_{C_{\delta_0}})\cdot f_n$ which has the desired properties.
\end{enumerate}

\exercise{3.4.7} In the forward direction, we suppose that $F$ is a
tempered distribution. The by Proposition 3.1.4 we can find an $N$
such that $|F(\varphi)| \leq C\norm{\varphi}_N$ for all $\varphi \in
\mathcal{S}$. Now suppose that $\varphi \in \mathcal{D}$ and has
supposrt in the ball $|x| \leq R$, then we have the bound $|x|^\beta \leq
c_{n,\beta}|x|^{|\beta|} \leq c_{n,N}R^n$. This implies that
\[
|F(\varphi)| = \sup_{\substack{x\in\R^d}}
|x^{\beta}\partial_{x}^{\alpha}\varphi| \leq
Cc_{n,N}R^N\sup_{\substack{x\in\R^d}} |\partial_x^\alpha\varphi|
\]
with $A = Cc_{n,N}$, as desired.

Conversely, suppose that $F$ is a functional that satisfies
\[
|F(\varphi)| \leq
AR^N\sup_{\substack{x\in\R^d}}|\partial_x^\alpha\varphi|
\]
for all $\varphi$ supported in $|x| \leq R$. Then we see that $F \in
\mathcal{D}^\ast$. Choose a partition of unity on $\R^d$, say $\psi_j$
with each of the $\psi_j$ supported in a ball about the origin. TO
construct such a partition we will use a function $\eta(x) + \eta(x-1)
= 1$ supported in $(-1,1)$. Then $\eta$ induces a partition of unity
given by $\psi_j(x) = \eta(|x| - j)$ supported in the disk of radius
$j$. For a given $\varphi_j \in \mathcal{D}$ so that
\[
\sum_{j=0}^\infty \psi_j\varphi = \varphi
\]
where the convergence is in $\mathcal{D}$. Thus, we can set
\[
|F(\varphi)| \leq |F(\psi_0\varphi)| + \sum_{j=1}^\infty
|F(\psi_j\varphi)|
\]
Becuase $\psi_0$ is supported in the unit ball, we take $R = 1$ to see
that
\[
|F(\psi_0\varphi)| \leq C\sum_{|\alpha|\leq N}
\sup_{x\in\R^n}|\partial^\alpha_x(\psi_0\varphi)| \leq
C'\sum_{|\alpha|\leq N}
\sup_{x\in\R^n}|\partial^\alpha_x(\varphi)|
\]
where $C'$ is a constant depending only on the $\psi_j$ and $N$. Now
we will go through a similar process for each of the
$|F(\psi_j\varphi)|$ yielding
\[
|F(\psi_j\varphi)| \leq C(j+1)^N \sum_{|\alpha|\leq N} \sup_{x \in
  B_j}|\partial_x^\alpha\varphi|
\]
As before we can then bound this with a new constant $C_j'$ such that
\begin{align*}
  C(j+1)^N \sum_{|\alpha|\leq N} \sup_{x \in
    \R^n}|\partial_x^\alpha\varphi| &\leq C_j'(j+1)^N
  \sum_{|\alpha|\leq N} \sup_{x \in B_j}|\partial_x^\alpha\varphi| \\
  &= C_j'(j+1)^{-2}\left(\frac{j+1}{j}\right)^{N+2} \sum_{|\alpha|
    \leq N} j^{N+2}|\partial_x^\alpha\varphi|
\end{align*}
Now we use the fact that $j \leq 1+|x|$ for $x$ in the ball of radius
$j$ and get a bound
\begin{align*}
  C_j'(j+1)^{-2}\left(\frac{j+1}{j}\right)^{N+2}\sum_{|\alpha| \leq N}
  j^{N+2}|\partial_x^\alpha\varphi| &\leq
  C_j'(j+1)^{-2}2^{N+2}\sum_{|\alpha| \leq N}
  (1+|x|)^{N+2}|\partial_x^\alpha\varphi| \\
  &\leq C_j'(j+1)^{-2}2^{N+2}\sum_{|\alpha| \leq N} \norm{\varphi}_{N+2}
\end{align*}
So we have bounded $F(\varphi)$ in terms of finite sums of
$\norm{\varphi}_N$ for various $N$. Hence, $F$ extends to be bounded on
$\mathcal{S}$ by density and so we have that $F \in \mathcal{S}^\ast$.

\exercise{3.4.8}

Suppose that $F$ is a homogeneous distribution of degree
$\lambda$. Then we know that $|F_a(\varphi)| = a^{\lambda}F$. Consider
a scaling operator $\eta \in \mathcal{D}$ with $\eta(x)$ in $|x| \leq
1$, supported in $|x| \leq 2$ and $\eta_R(x) = \eta(x/R)$. We know
because $F$ is a distribution and $\eta$ has compact support
\[
|\eta_1F(\varphi)| = |f(\eta_1\varphi)| \leq C \sum_{|alpha| \leq N}
\sup_{x \leq 1} |\partial_x^\alpha\varphi| \leq C\norm{\varphi}_N
\]
We can see this because we can approximate $F$ by a sequence of
distributions with compact sense $F_n \to F$ in the weak sense as $n
\to \infty$ and by the discussion after Proposition 3.1.4, the limit
holds for each of the $F_n$ and thus for $F$ as well. Now suppose that
we have a distribution $\varphi \in \mathcal{D}$ that is supported in
the ball of radius $R$ about the origin with $R \geq 1$. Then we see
that $\eta_R\varphi$ is supported in the unit ball and the above estimate
gives
\begin{align*}
  |\eta_RF(\varphi)| &\leq C \sum_{|\alpha| \leq N} \sup_{|x| \leq 1}
  |\partial_x^\alpha(\eta_R\varphi)| \\
  &= C \sum_{|\alpha| \leq N} \sup_{|x| \leq 1}
  R^{-|\alpha|}|\partial_x^\alpha(\varphi)|
\end{align*}
Now because $F$ is homogenous we also know that $F(\eta_R\varphi) =
R^{-n - \lambda}|F(\varphi)|$. Combining these facts we see that
\[
|F(\eta_R\varphi)| \leq CR^{n+\lambda}\sum_{|\alpha| \leq N}
\sup_{|x|\leq R} |\partial_x^\alpha\varphi| \leq
CR^{n+\lambda}\norm{\varphi}_N
\]
We can then apply the previous exercise to see that $F$ must be tempered.

\exercise{3.4.10}

To see that $\mathcal{D}$ is dense in $\mathcal{S}$ we let $\varphi
\in S$ fix an $\eta \in \mathcal{D}$ that has $\eta = 1$ on the unit
ball. We then set $\eta_k(x) = \eta(x/k)$ and $\varphi_k =
\eta_k\varphi$. Then consider $\norm{\varphi_k - \varphi}_N$,
\begin{align*}
  \norm{\varphi_k - \varphi}_N &= \sup_{\substack{x\in \R^d \\
      |\alpha|,|\beta| \leq N}}
  |x^\beta\partial_x^\alpha(\eta_k\varphi - \varphi)| \\
  &= \sup_{\substack{x\in \R^d \\
      |\alpha|,|\beta| \leq N}}
  |x^\beta\partial_x^\alpha(\eta_k\varphi) -
  x^\beta\partial_x^\alpha(\varphi)|
\end{align*}
Then we compute
\begin{align*}
  \partial_x^\alpha(\eta_k\varphi) &= \sum_{|\gamma| \leq |\alpha|}
  \binom{|\alpha|}{|\gamma|}k^{|\gamma - \alpha|}\partial_x^\gamma
  \varphi\partial_x^{\gamma - \alpha}\eta_k \\
  &= \partial_x^\alpha\varphi +\sum_{\substack{|\gamma| \leq |\alpha|
      \\ \gamma \neq \alpha}} \binom{\alpha}{\gamma}k^{|\gamma -
    \alpha|}\partial_x^\gamma \varphi\partial_x^{\gamma -
    \alpha}\eta_k
\end{align*}
So we can estimate the above by
\begin{align*}
  \sup_{\substack{x\in \R^d \\
      |\alpha|,|\beta| \leq N}}
  |x^\beta\partial_x^\alpha(\eta_k\varphi) -
  x^\beta\partial_x^\alpha(\varphi)| &\leq \left|
    x^{\beta}\sum_{\substack{|\gamma| \leq |\alpha| \\ |\gamma -
        \alpha| \geq 1}} \binom{|\alpha|}{|\gamma|}k^{|\gamma -
      \alpha|}\partial_x^\gamma \varphi\partial_x^{\gamma -
      \alpha}\eta_k\right| \\
  &\leq \frac{\norm{\varphi}_N}{k} \sum_{\substack{|\gamma| \leq
      |\alpha| \\ |\gamma - \alpha| \geq 1}} \norm{\varphi}_{|\alpha|}
\end{align*}
This clearly goes to zero as $k \to \infty$ for all $N$ and so we have
that $\varphi_k \to \varphi$ as desired.

\exercise{3.4.11}
\begin{enumerate}
\item Suppose that both $\varphi_1,\varphi \in \mathcal{S}$. Then we
  need to show that
  \[
  \norm{\varphi_1\varphi_2}_N = \sup_{\substack{x\in\R^d \\
      |\alpha|,|\beta| \leq N}}
  |x^\beta\partial_x^\alpha(\varphi_1\varphi_2)| <\infty
  \]
  We then compute the product via
  \begin{align*}
    \partial_x^\alpha(\varphi_1\varphi_2) &= \sum_{\gamma \leq
      \alpha}\binom{\alpha}{\gamma}(\partial_x^{\alpha-\gamma}\varphi_1)
    (\partial_x^\gamma \varphi_2)
  \end{align*}
  Substituting this formula above gives
  \begin{align*}
    \sup_{\substack{x\in\R^d \\
        |\alpha|,|\beta| \leq N}}
    |x^\beta\partial_x^\alpha(\varphi_1\varphi_2)| &=
    \sup_{\substack{x\in\R^d \\
        |\alpha|,|\beta| \leq N}} \left|x^{\beta}\sum_{\gamma \leq
        \alpha}\binom{\alpha}{\gamma}(\partial_x^{\alpha-\gamma}\varphi_1)
      (\partial_x^\gamma \varphi_2)\right| \\
    &\leq \sup_{\substack{x\in\R^d \\
        |\alpha|,|\beta| \leq N}}\left|\sum_{\gamma \leq
        \alpha}\binom{\alpha}{\gamma}\left(x^{N -
          |\gamma|}\partial_x^{\alpha-\gamma}\varphi_1\right)
      \left(x^{|\gamma|}\partial_x^\gamma \varphi_2\right)\right| \\
    &\leq \sup_{\substack{x\in\R^d \\
        |\alpha|,|\beta| \leq N}} \sum_{\gamma \leq
      \alpha}\binom{\alpha}{\gamma}\norm{\varphi_1}_{N-|\gamma|}
    \norm{\varphi_{2}}_{|\gamma|}
  \end{align*}
  This is a finite sum and hence, is bounded so that
  $\norm{\varphi_1\varphi_2}_N \leq \infty$ and so $\varphi_1\varphi_2
  \in \mathcal{S}$ as claimed.
\item To see that $\varphi_1 \ast \varphi_2 \in \mathcal{S}$ whenever
  $\varphi_1, \varphi_2 \in \mathcal{S}$ we will invoke the Fourier
  transform. We saw that the Fourier transform $\mathcal{F}:
  \mathcal{S} \to \mathcal{S}$ was a homeomorphism and that
  \[
  \mathcal{F}(\varphi_1 \ast \varphi_2) =
  \mathcal{F}(\varphi_1)\mathcal{F}(\varphi_2)
  \]
  Because $\varphi_1$ and $\varphi_2$ are both in $\mathcal{S}$ we
  conclude that $\mathcal{F}(\varphi_1 \ast \varphi_2) \in
  \mathcal{S}$ as well. Then applying Fourier inversion in
  $\mathcal{S}$ we get that $\varphi_1\ast\varphi_2 \in \mathcal{S}$
  as desired.
\item To verify that $\varphi_1 \ast \varphi_2 \in \mathcal{S}$
  directly from the definition we first recall that differentiation
  commutes with the convolution to compute
  \begin{align*}
    \norm{\varphi_1\ast \varphi_2}_N &= \sup_{\substack{x\in\R^d \\
        |\alpha|,|\beta| \leq N}}
    |x^{\beta}\partial_x^\alpha(\varphi_1\ast\varphi_2)| \\
    &= \sup_{\substack{x\in\R^d \\
        |\alpha|,|\beta| \leq N}}\left|x^\beta \varphi_1
      \ast \partial_{x-y}^\alpha\varphi_2\right| \\
    &\leq \sup_{\substack{x\in\R^d \\
        |\alpha|,|\beta| \leq N}} \int_{\R^d}
    |\varphi_1(y)|\cdot|x^\beta\partial_{x-y}^\alpha\varphi(x-y)|dy \\
    &\leq \sup_{\substack{x\in\R^d \\
        |\alpha|,|\beta| \leq N}} \sum_{\gamma \leq
      \beta}\binom{\beta}{\gamma}\int_{\R^d} |y^\ell\varphi_1(y)|\cdot
    |(x-y)^{\beta-\ell}\partial_{x-y}^\alpha\varphi(x-y)|dy \\
    &\leq \sup_{\substack{x\in\R^d \\
        |\alpha|,|\beta| \leq N}} \sum_{\gamma \leq
      \beta}\binom{\beta}{\gamma}\norm{\varphi_2}_N\int_{\R^d}
    |y^\ell\varphi_1(y)| \frac{1+y^2}{1+y^2}dy \\
    &\leq \sup_{\substack{x\in\R^d \\
        |\alpha|,|\beta| \leq N}} C\sum_{\gamma \leq
      \beta}\binom{\beta}{\gamma}\norm{\varphi_2}_N
    (\norm{\varphi_1}_{\gamma} + \norm{\varphi}_{|\gamma|+2})
  \end{align*}
  The last term is bounded because it is a finite sum of bounded
  terms. Hence $\varphi_1\ast\varphi_2 \in \mathcal{S}$
\end{enumerate}

\exercise{3.4.12} Let $F$ be a distribution with compact support and
let $\varphi \in \mathcal{S}$. We will first show that for each $N$ we
can find a constant $C_N$ such that
\[
\norm{\varphi_y^\sim}_N \leq C_N(1+|y|)^N\norm{\varphi}_N
\]
Indeed, we first note that
\[
\norm{\varphi_y^\sim}_N = \sup_{\substack{y \in \R^d \\
    |\alpha|,|\beta| \leq N}}|y^\beta\partial_y^\alpha\varphi(x-y)
\leq \sup_{\substack{y \in \R^d \\
    |\alpha|,|\beta| \leq
    N}}|(1+|y|)^\beta\partial_y^\alpha\varphi(y-y)
\]
Then we note that $(1+|y|) \leq (1+|y|)(1+|x-y|)$ to conclude that
\begin{align*}
  \sup_{\substack{y \in \R^d \\
      |\alpha|,|\beta| \leq
      N}}|(1+|y|)^\beta\partial_y^\alpha\varphi(x-y)| &\leq \sup_{
    \substack{y \in \R^d \\
      |\alpha|,|\beta| \leq
      N}}|(1+|y|)^\beta(1+|x-y|)^\beta\partial_y^\alpha\varphi(x-y)| \\
  &\leq C_N(1+|y|)^N\sup_{\substack{y\in \R^d \\ |\alpha|,|\beta|\leq
      N}} ||x-y|^\beta\partial_y^\alpha(x-y)| \\
  &\leq C_N(1+|y|)^N\norm{\varphi}_N
\end{align*}
Now we note that because $F$ has compact support it must also be
tempered and therefore we can apply Proposition 3.1.4 to see that
the function $F\ast \varphi = F(\varphi_y^\sim)$ must satisfy
\[
|F(\varphi_x^\sim)| \leq c\norm{\varphi_y^\sim}_N \leq
C'(1+|x|)^N\norm{\varphi}_N
\]
Thus, we see that
\begin{align*}
  \norm{F(\varphi_x^\sim}_N = \sup_{\substack{x \in \R^d \\
      |\alpha|,|\beta|\leq N}} |x^{\beta}\partial_x^\alpha
\end{align*}
\end{document}