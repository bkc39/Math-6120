\documentclass{article}
\usepackage[tmargin=1in,bmargin=1in,lmargin=1.5in,rmargin=1.5in]{geometry}
\usepackage{amsfonts,amsmath,amssymb,amsthm}
\usepackage{mathrsfs}
\usepackage{ccfonts}
\usepackage{relsize,fancyhdr,parskip}
\usepackage{graphicx}

\usepackage{tikz}
\usetikzlibrary{shapes, shapes.geometric, shapes.symbols,
  shapes.arrows, shapes.multipart, shapes.callouts, shapes.misc}

\pagestyle{fancy}
\lhead{Ben Carriel}
\chead{Math 6120 Problem Set 2}
\rhead{\today}

\parskip 7.2pt
\parindent 8pt

\newcommand{\tab}{\hspace*{2em}}
\newcommand{\tand}{\tab\text{and}\tab}

\DeclareMathOperator{\N}{\mathbb{N}}
\DeclareMathOperator{\Z}{\mathbb{Z}}
\DeclareMathOperator{\Q}{\mathbb{Q}}
\DeclareMathOperator{\R}{\mathbb{R}}
\DeclareMathOperator{\C}{\mathbb{C}}
\DeclareMathOperator{\D}{\mathbb{D}}
\DeclareMathOperator{\capchi}{\raisebox{2pt}{$\mathlarger{\mathlarger{\chi}}$}}

\DeclareMathOperator{\divides}{\mathrel{|}}
\DeclareMathOperator{\suchthat}{\mathrel{:}}

\DeclareMathOperator{\lra}{\longrightarrow}
\DeclareMathOperator{\into}{\hookrightarrow}
\DeclareMathOperator{\onto}{\twoheadrightarrow}
\DeclareMathOperator{\bijection}{\leftrightarrow}
\DeclareMathOperator{\lap}{\bigtriangleup}

\newcommand{\problem}[1]{\noindent{\textbf{Problem #1}}\\}
\newcommand{\problempart}[1]{\noindent{\textbf{(#1)}}}
\newcommand{\exercise}[1]{\noindent{\textbf{Exercise #1:}}}

\newcommand{\der}[2]{\frac{\partial #1}{\partial #2}}
\newcommand{\norm}[1]{\|#1\|}
\newcommand{\diam}[1]{\text{diam}(#1)}
\newcommand{\seq}[2]{\{#1_{#2}\}_{#2 = 1}^\infty}

\newcommand{\conj}[1]{\overline{#1}}
\newcommand{\cis}[1]{\operatorname{cis}#1}


\newcommand{\real}{\mathrel{\text{Re}}}
\newcommand{\imag}{\mathrel{\text{Im}}}
\newtheorem*{thm}{\\ Theorem}
\newtheorem*{lem}{\\ Lemma}
\newtheorem*{claim}{\\ Claim}
\newtheorem*{defn}{\\ Definition}
\newtheorem*{prop}{\\ Proposition}

\begin{document}

\exercise{2.6.2}

We want to show
\[
\int_0^\infty \frac{\sin x}{x}dx = \frac{\pi}{2}
\]
Consider the integral
\[
\frac{1}{2i}\int_{-\infty}^\infty \frac{e^{ix}}{x}dx =
\frac{1}{2i}\int_{-\infty}^\infty \frac{i\sin x}{x}dx +
\int_{-\infty}^\infty \frac{\cos x}{x}
\]
Because $\cos x$ is even and $\sin x$ is odd we see
\[
\frac{1}{2i}\int_{-\infty}^\infty \frac{i\sin x}{x}dx +
\frac{1}{2i}\int_{-\infty}^\infty \frac{\cos x}{x} =
\frac{1}{2i}\int_{-\infty}^\infty \frac{i\sin x}{x} = \int_0^\infty
\frac{\sin x }{x}
\]
We will evaluate this on the contour

\begin{figure}[h!]
  \centering
  \begin{tikzpicture}
    \node[below] (O) at (0,0) {};
    \node[below] (LR) at (-1.5,0) {$-R$};
    \node[below] (RR) at (1.5,0) {$R$};
    \node[below] (LE) at (-.4,0) {$-\epsilon$};
    \node[below] (RE) at (.4,0) {$+\epsilon$};
    \node[above] (GR) at (1,1.3) {$\gamma_R^+$};
    \node[above] (GE) at (.3,.3) {$\gamma_\epsilon^+$};
    \draw[->] (1.5,0) arc (0:90:1.5);
    \draw (0,1.5) arc (90:180:1.5);
    \draw (.4,0) arc (0:90:.4);
    \draw[<-] (0,.4) arc (90:180:.4);
    \draw[->] (-1.5,0) -- (-.95,0);
    \draw (-.95,0) -- (-.4,0);
    \draw[->] (.4,0) -- (.95,0);
    \draw (.95,0) -- (1.5,0);
    \draw (-.4,0) -- (-.15,0);
    \draw[<->] (-.15,0) -- (.15,0);
    \draw (.15,0) -- (.4,0);
  \end{tikzpicture}
  \caption{Contour of integration}
  \label{fig:1}
\end{figure}

We apply Cauchy's theorem to see that
\[
\int_{\gamma_R^+} \frac{e^{iz}}{z}dz + \int_{-R}^{-\epsilon}
\frac{e^{ix}}{x}dx + \int_{\gamma_R^+} \frac{e^{iz}}{z}dz +
\int_{\epsilon}^{R} \frac{e^{ix}}{x}dx = 0
\]
Then we observe
\[
\int_{\gamma_R^+} \frac{e^{iz}-1}{z}dz =
\int_{\gamma_R^+}\frac{e^{iz}}{z}dz \leq
\left|\int_{\gamma_R^+}\frac{e^{iz}}{z}dz\right|
\]
We estimate the last term
\[
\left|\int_{\gamma_R^+}\frac{e^{iz}}{z}dz\right| = \left|\int_{0}^\pi
  \frac{e^{iRe^{i\theta}}}{Re^{i\theta}}iRe^{i\theta}d\theta\right|
\leq \left|i\int_0^\pi e^{iR\cos\theta}e^{-R\sin \theta}d\theta\right|
\]
Now estimate
\[
\left|i\int_0^\pi e^{iR\cos\theta}e^{-R\sin \theta}d\theta\right| \leq
\left|i\int_0^\pi e^{-R\sin\theta}d\theta\right|
\]
Note that if $\theta \in (0,\pi)$ then $e^{\sin\theta} > 1$, which can
be seen by observing that $\sin\theta > 0$ in $(0,\pi)$ and then
applying the esponential function to both sides of the inequality. So
\[
\left|i\int_0^\pi e^{-R\sin\theta}d\theta\right| \leq i\pi
\sup_{\theta \in (0,\pi)}\left|(e^{\sin\theta})^{-R}\right| \leq i\pi
(1+\epsilon)^{-R}
\]
Letting $R\to \infty$ shows that the integral goes to $0$. This means
\[
\int_{-R}^{-\epsilon} \frac{e^{ix}}{x}dx + \int_{\epsilon}^{R}
\frac{e^{ix}}{x}dx = - \int_{\gamma_\epsilon^+} \frac{e^{iz}}{z}dz
\]
Now observe that
\[
\frac{e^{iz}}{z} = \frac{\sum_{n=1}^\infty\frac{(iz)^n}{n!}}{z} =
\frac{1 + O(z)}{z} = \frac{1}{z} + O(1)
\]
So that we have
\[
-\int_{\gamma_\epsilon^+} \frac{e^{iz}}{z}dz \approx
-\int_{\gamma_\epsilon^+} \frac{1}{z}dz = -\int_\pi^0 \frac{i\epsilon
  e^{i\theta}}{\epsilon e^{i\theta}}d\theta = \int_0^\pi id\theta =
\pi i
\]
Letting $\epsilon \to 0$ and $R \to \infty$ gives that
\[
\int_{-\infty}^\infty \frac{\sin x}{x}dx = \pi i
\]
We then apply the original identity (dividing by $2i$) to see that
\[
\int_{0}^\infty \frac{\sin x}{x}dx = \frac{\pi}{2}
\]
{
\footnotesize{Note: I think the hint in the book is a typo because
 the extra $-1$ in the numerator interferes with the limiting computation.
}}

\exercise{2.6.6}

There are two cases
\begin{enumerate}
\item $w$ is in the interior of $T$.
\item $w$ is on the boundary of $T$.
\end{enumerate}
We treat both individually. Suppose that $w$ is an interior point of
$T$. Then we will integrate around a ``keyhole'' toy contour as
outlined on page 42 of the text. In this case we let the width of the
``corridor'' be $\epsilon$ for any choice of $\epsilon >
0$. Consequently, we can apply Cauchy's theorem to see that for any
$\epsilon > 0$ the integral about the contour is zero. So we only need
to deal with the small disk that forms around $w$. Because $f$ is
bounded near $w$ we can make the estimate
\[
\left| \int_{\gamma_\epsilon} f(z)dz\right| \leq
\int_{\gamma_\epsilon}|f(z)|dz \leq \int_{0}^{2\pi} M\epsilon
e^{i\theta}d\theta \leq \epsilon 2M\pi
\]
So the integral around the inner loop goes to zero as $\epsilon \to 0$.

If $w$ is on the boundary of $T$. Then we choose a slightly different
contour: instead of a ``corridor'' we just use a semicircle around
$w$. This is analogous to the the construction on page 44 of the text,
except the contour is cut out of a triangle instead of a larger
semicircle. Again, we apply Cauchy's theorem to see that the integral
around the whole contour is zero as $\epsilon \to 0$ by estimating
again. We make the same estimate as before
\[
\left| \int_{\gamma_\epsilon} f(z)dz\right| \leq
\int_{\gamma_\epsilon}|f(z)|dz \leq \int_{0}^{\pi} M\epsilon
e^{i\theta}d\theta \leq \epsilon M\pi
\]
Which goes to zero. Hence, the integral around the entire contour is zero.

So the claim holds in both cases, and therefore is $w$ is any element
of $T$ and $f$ is bounded in a neighbor of $w$ then
\[
\int_T f(z)dz = 0
\]

\exercise{2.6.11}

\begin{enumerate}
\item[\textbf{(a)}] We need to show that when $0 < R < R_0$ and $|z| <
  R$ that
\[
f(z) = \frac{1}{2\pi}\int_0^{2\pi}
f(Re^{i\varphi})\real\left(\frac{Re^{i\varphi} +
    z}{Re^{i\varphi}-z}\right)d\varphi
\]
We begin with the right hand side, multiplying by the conjugate to
separate and get
\begin{align*}
  \frac{1}{2\pi}\int_0^{2\pi}
  f(Re^{i\varphi})\real\left(\frac{Re^{i\varphi} +
      z}{Re^{i\varphi}-z}\right)d\varphi &=
  \frac{1}{4\pi}\int_0^{2\pi}
  f(Re^{i\varphi})\real\left(\frac{Re^{i\varphi} + z}{Re^{i\varphi}-z}
    + \frac{Re^{-i\varphi} +
      \bar{z}}{Re^{-i\varphi}-\bar{z}}\right)d\varphi \\
  &= \frac{1}{4\pi}\int_0^{2\pi}
  f(Re^{i\varphi})\real\left(2\frac{Re^{i\varphi}}{Re^{i\varphi}-z} -
    \frac{2\bar{z}}{\bar{z} - Re^{-i\varphi}}\right)d\varphi
\end{align*}
Now we distribute and break up the computation as follows
\begin{align*}
  &\frac{1}{4\pi}\int_0^{2\pi}
  f(Re^{i\varphi})\real\left(2\frac{Re^{i\varphi}}{Re^{i\varphi}-z} -
    \frac{2\bar{z}}{\bar{z} - Re^{-i\varphi}}\right)d\varphi\\ &=
  \frac{1}{2\pi i}\int_0^{2\pi}
  f(Re^{i\varphi})\frac{iRe^{i\varphi}}{Re^{i\varphi}-z}d\varphi +
  \frac{1}{2\pi}\int_0^{2\pi}
  f(Re^{i\varphi})\frac{iRe^{i\varphi}}{Re^{i\varphi} - R^2\bar{z}^{-1}}d\varphi
\end{align*}
Now observe that we can apply the Cauchy integral formula to see
\[
f(z) = \frac{1}{2\pi i}\int_0^{2\pi}
  f(Re^{i\varphi})\frac{iRe^{i\varphi}}{Re^{i\varphi}-z}d\varphi
\]
Then we note that
\[
\frac{1}{2\pi}\int_0^{2\pi}
f(Re^{i\varphi})\frac{iRe^{i\varphi}}{Re^{i\varphi} -
  R^2\bar{z}^{-1}}d\varphi = 0
\]
is an analytic function on the interior of $|z| < R$

\item[\textbf{(b)}] We begin by computing
\[
\frac{Re^{i\gamma}+r}{Re^{i\gamma}-r} = \frac{R\cos\gamma +
  iR\sin\gamma + r}{R\cos\gamma + iR\sin\gamma - r}
\]
Now rationalize the denominator to get
\[
\frac{R\cos\gamma + iR\sin\gamma + r}{R\cos\gamma + iR\sin\gamma - r}
= \frac{(R\cos\gamma + iR\sin\gamma + r)(R\cos\gamma + iR\sin\gamma -
  r)}{(R\cos\gamma - r)^2 + R^2\sin^2\gamma}
\]
We do the multiplication to see
\[
\frac{(R\cos\gamma + iR\sin\gamma + r)(R\cos\gamma + iR\sin\gamma -
  r)}{(R\cos\gamma - r)^2 + R^2\sin^2\gamma} = \frac{(R^2\cos^2\gamma
  + R^2\sin^2\gamma - r^2)-i(2Rr\sin\gamma)}{R^2 -2Rr\cos\gamma + r^2}
\]
Taking the real part confirms that
\[
\real\left(\frac{Re^{i\gamma}+r}{Re^{i\gamma}-r}\right) = \frac{R^2 -
  r^2}{R^2 -2Rr\cos\gamma + r^2}
\]
\end{enumerate}

\exercise{2.6.12}
\begin{enumerate}
\item[\textbf{(a)}] Following the hint, we recall that $f'(z) =
2\partial{u}/\partial{z}$. Consider $g(z) =
2\partial{u}{z}$. We want to show that $g$ is holomorphic. Indeed,
observe that
\[
\der{g}{\bar{z}} = 2\der{u}{\bar{z}}\der{u}{z} = 0
\]
by the equality of mixed partials. Hence, $g$ is holomorphic.

Now we apply Theorem 2.2.1 to see that $g$ must have a primitive, say
$f$. Then we compute
\[
\der{\mathrel{\text{Re}}(f)}{z} = \frac{1}{2}\der{f}{z} = \frac{1}{2}g
= \der{u}{z}
\]
As a result, $\mathrel{\text{Re}}(f)$ differs from $u$ by some real
constant $k$. Hence, $f(z)$ is the desired function.

\item[\textbf{(b)}] We apply the previous exercise (11) to the function
\[
f(z) = u(z) + iv(z)
\]
So we get
\[
f(z) = \frac{1}{2\pi}\int_0^{2\pi} (u(e^{i\theta} +
iv(e^{i\theta}))\mathrel{\text{Re}}\left(\frac{e^{i\theta}+z}
  {e^{i\theta}-z}\right)d\theta
\]
Now we apply \textbf{(a)} to separate the real part and see
\begin{align*}
  u(re^{i\varphi}) &= \frac{1}{2\pi}\int_0^{2\pi}
  u(e^{i\theta})\mathrel{\text{Re}}\left(\frac{e^{i\theta} +
      re^{i\varphi}}{e^{i\theta} - re^{i\varphi}}\right)d\theta \\
  &= \frac{1}{2\pi}\int_0^{2\pi}
  u(e^{i\theta})\mathrel{\text{Re}}\left(\frac{e^{i(\theta - \varphi)
      } +
      r}{e^{i(\theta -\varphi}- r}\right)d\theta \\
  &= \frac{1}{2\pi}\int_0^{2\pi}
  u(e^{i\theta})\frac{1-r^2}{1-2r\cos(\varphi - \theta)+r^2}d\theta\\
  &= \frac{1}{2\pi}\int_0^{2\pi} P_r(\varphi - \theta)u(e^{i\theta)})d\theta
\end{align*}
as desired.
\end{enumerate}

\end{document}
