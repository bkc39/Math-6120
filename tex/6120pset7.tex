\documentclass{article}
\usepackage[tmargin=1in,bmargin=1in,lmargin=1.5in,rmargin=1.5in]{geometry}
\usepackage{amsfonts,amsmath,amssymb,amsthm}
\usepackage{mathrsfs}
\usepackage{ccfonts}
\usepackage{relsize,fancyhdr,parskip}
\usepackage{graphicx}

\usepackage{tikz} \usetikzlibrary{shapes, shapes.geometric,
  shapes.symbols, shapes.arrows, shapes.multipart, shapes.callouts,
  shapes.misc,decorations.markings,decorations.shapes}

\pagestyle{fancy}
\lhead{Ben Carriel}
\chead{Math 6120 Problem Set 5}
\rhead{\today}

\headheight 13.0pt
\parskip 7.2pt
\parindent 8pt

\newcommand{\tab}{\hspace*{2em}}
\newcommand{\tand}{\tab\text{and}\tab}

\DeclareMathOperator{\N}{\mathbb{N}}
\DeclareMathOperator{\Z}{\mathbb{Z}}
\DeclareMathOperator{\Q}{\mathbb{Q}}
\DeclareMathOperator{\R}{\mathbb{R}}
\DeclareMathOperator{\C}{\mathbb{C}}
\DeclareMathOperator{\Ha}{\mathbb{H}}
\DeclareMathOperator{\D}{\mathbb{D}}
\DeclareMathOperator{\T}{\mathbb{T}}
\DeclareMathOperator{\capchi}{\raisebox{2pt}{$\mathlarger{\mathlarger{\chi}}$}}

\DeclareMathOperator{\divides}{\mathrel{|}}
\DeclareMathOperator{\suchthat}{\mathrel{:}}

\DeclareMathOperator{\lra}{\longrightarrow}
\DeclareMathOperator{\into}{\hookrightarrow}
\DeclareMathOperator{\onto}{\twoheadrightarrow}
\DeclareMathOperator{\bijection}{\leftrightarrow}
\DeclareMathOperator{\lap}{\bigtriangleup}

\newcommand{\problem}[1]{\noindent{\textbf{Problem #1}}\\}
\newcommand{\problempart}[1]{\noindent{\textbf{(#1)}}}
\newcommand{\exercise}[1]{\noindent{\textbf{Exercise #1:}}}

\newcommand{\der}[2]{\frac{\partial #1}{\partial #2}}
\newcommand{\norm}[1]{\|#1\|}
\newcommand{\diam}[1]{\text{diam}(#1)}
\newcommand{\seq}[2]{\{#1_{#2}\}_{#2 = 1}^\infty}

\newcommand{\conj}[1]{\overline{#1}}
\newcommand{\cis}[1]{\operatorname{cis}#1}
\newcommand{\res}{\operatorname{res}}

\newcommand{\real}{\mathrel{\text{Re}}}
\newcommand{\imag}{\mathrel{\text{Im}}}
\newtheorem*{thm}{\\ Theorem}
\newtheorem*{lem}{\\ Lemma}
\newtheorem*{claim}{\\ Claim}
\newtheorem*{defn}{\\ Definition}
\newtheorem*{prop}{\\ Proposition}

\begin{document}
\exercise{8.5.5}

First note that $f(z) = -\frac{1}{2}(z + 1/z)$ is holomorphic in the
half disk $U = \{z = x+iy \suchthat |z| < 1, y>1\}$ because the origin
is excluded. We will first prove that $f$ is surjective from $U \to
\Ha$. Indeed, we note that if $f(z) = w$ then $z^2+2wz+1 = 0$. This
equation has two distinct roots in $\C$ if $w \neq \pm 1$, which is
true because we exclude the real line in $\Ha$. Moreover, we see that
one of the roots lies in the interior of $U$ because the roots
are of the form $z_0 = -w \pm (w^2-1)$ and the product of the roots
must be 1. So if we note that
\[
|z_0| = |-w - (w^2-1)| = |w+w^2+1| > 1, w \in \Ha
\]
Then the other root must satisfy $|z_1| = |1/w| < 1$ and lies in
$U$. Then $f(z_1) = w$ and $f$ is surjective.

For injectivity, we suppose that there were two distinct $z_0,z_1 \in
U$ such that $w = f(z_0) = f(z_1)$. We again use the fact that
$z_0,z_1$ must be the roots of the equation $z^2+2wz+1$ and by the
above we see that only one of them can be in $U$. So $f$ is injective
on $U$.

Finally, we check that $f$ maps into $\Ha$. This is clear because if $z
\in U$ then
\[
\imag{(-\frac{1}{2}(z+1/z))} = \frac{-1}{2}(\imag(z) -
\frac{1}{|z|}\imag(z)) = -\frac{1}{2}\left(1 - \frac{1}{|z|}\right)\imag(z)
\]
Which is greater than zero when $|z| < 1$ (i.e. $z \in U$). So $f$
maps $U$ into $\Ha$.

\exercise{8.5.6}

We will verify this with a direct calculation. We let $u:U\to\C$ be a
harmonic function and $F = f(x,y) + ig(x,y)$ be holomorphic from $V\to
U$. We then use the fact that
\[
\der{^2}{z\conj{z}} = \frac{1}{4}\Delta
\]
to compute
\begin{align*}
  \Delta(u \circ F) &= \frac{1}{4}(u \circ F)_{z\conj{z}} \\
  &= \frac{1}{4}((u_z \circ F)\cdot F')_{\conj{z}} \\
  &= \frac{1}{4}(u_{z\conj{z}} \circ F)\cdot F'\cdot\conj{F'} \\
  &= (\Delta u\circ F)\cdot |F'|^2
\end{align*}
And because $\Delta u = 0$ the entire expresion is zero and we see
that
\[
\Delta(u \circ F) = 0
\]
as well. Hence, $u \circ F$ is harmonic.

\exercise{8.5.12}
\begin{enumerate}
\item[\textbf{(a)}] Suppose that $f:\D \to \D$ is analytic and has two
  fixed points $z_0,z_1$. Then we recall the function
  \[
  \psi_{z_0}(z) = \frac{z-z_0}{1-\conj{z_0}z}
  \]
  from the text, whose important property is that $\psi_{z_0}(0) =
  z_0$ and $\psi_{z_0}(z_0) = 0$. Then the function
  \[
  F = \psi_{z_0}^{-1} \circ f \circ \psi_{z_0}
  \]
  fixes the origin. Now we apply the Schwarz lemma to see that $|F(z)|
  \leq |z|$ for $|z|\in \D$. Moreover, because equality holds, then
  $F(z) = e^{i\theta}z$ for some $\theta$. Because $F(b) = b$ we see
  that $\theta = 0$. As a result
  \[
  F(z) = (\psi_{z_0}^{-1} \circ f \circ \psi_{z_0})(z) = z
  \]
  And hence, $f(z) = z$ for every $z$ and $f$ is the identity.
\item[\textbf{(b)}] Following the hint, we look at maps in
  $\Ha$. Consider the horizontal translations in the upper half plane
  of the form $\varphi_n: z \mapsto z+n$ for some integer $n$. Now take
  any conformal map $f: \D \to \Ha$ and consider $F = \varphi_n^{-1}
  \circ f \circ \varphi_n$. Then $F: \D \to \D$ is holomorphic, but
  has no fixed points.
\end{enumerate}

\exercise{8.5.13}
\begin{enumerate}
\item[\textbf{(b)}] We are considering the hyperbolic distance defined
  as
  \[
  \rho(z,w) = \left|\frac{z-w}{1-\conj{w}z}\right|
  \]
  By part \textbf{(a)} we have that if $f \in \text{Aut}(\D)$ then
  \begin{eqnarray}
    \rho(f(z),f(w)) &\leq& \rho(z,w) \nonumber \\
    \left|\frac{f(z) - f(z)}{1 - \conj{f(w)}f(z)}\right| &\leq&
    \left|\frac{z-w}{1-\conj{w}z}\right| \nonumber \\
    \left|\frac{f(z) - f(z)}{z-w}\right|\cdot\frac{1}{|1 - \conj{f(w)}f(z)|
    }
    &\leq& \frac{1}{|1-\conj{w}{z}|} \nonumber
  \end{eqnarray}
  Now we let $w\to z$ and get the estimate
  \[
  \frac{|f'(z)|}{1-|f(z)|^2} \leq \frac{1}{1-|z|^2}
  \]
  for all $z \in \D$.
\end{enumerate}

\exercise{8.5.14}

Let $f: \Ha \to \D$ be a conformal map. Recall from the text the map
\[
G(w) = i\frac{1-w}{1+w}
\]
which is conformal from $\D\to\Ha$. Then the composition of functions
$f \circ G: \D \to \D$ is holomorphic and furthermore it is an
automorphism of $\D$. As a result, we can apply Theorem 8.2.2 to see that
\[
(f \circ G)(z) = e^{i\theta}\frac{\alpha - z}{2-\conj{\alpha}z}
\]
or some $\theta \in \R$ and $\alpha\in \D$. Then we simplify by
computing setting $z = G(w)$ and getting
\[
f(z) = f\left(i\frac{1-w}{1+w}\right)=e^{i\theta}\frac{\alpha -
  w}{1-\conj{\alpha}w}
\]
Then recall that $G$ has an inverse map
\[
F(z) = \frac{i-z}{i+z}
\]
and do $w = F(z)$ and we can compute
\[
e^{i\theta}\frac{\alpha - w}{1-\conj{\alpha}w} =
e^{i\theta}\frac{\alpha -
  \frac{i-z}{i+z}}{1-\conj{\alpha}\frac{i-z}{i+z}} =
e^{i\theta}\frac{(i+z)\alpha - (i-z)}{(i+z)-\conj{\alpha}(i-z)}
\]
Rearranging terms gives
\[
e^{i\theta}\frac{(i+z)\alpha - (i-z)}{(i+z)-\conj{\alpha}(i-z)} =
e^{i\theta}\frac{z(1+\alpha) -
  i(1-\alpha)}{z(1+\alpha)-i(1-\conj{\alpha})} = e^{i\theta}\frac{z -
  i\frac{1-\alpha}{1+\alpha}}{z + i\frac{1-\conj{\alpha}}{1+\alpha}}
\]
We then make the substitution
\[
\beta = i\frac{1-\alpha}{1+\alpha}
\]
in the above to get
\[
e^{i\theta}\frac{z(1+\alpha) -
  i(1-\alpha)}{z(1+\alpha)-i(1-\conj{\alpha})} = e^{i\theta}\frac{z -
  i\frac{1-\alpha}{1+\alpha}}{z + i\frac{1-\conj{\alpha}}{1+\alpha}} =
e^{i\theta}\frac{z-\beta}{z-\beta}
\]
as desired.
\end{document}